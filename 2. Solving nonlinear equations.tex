\documentclass[a4paper, 12pt]{article}
\usepackage[utf8]{inputenc}
\usepackage[english, ukrainian]{babel}
\usepackage{amsmath, amssymb}
\usepackage[top = 1 cm, left = 1 cm, right = 1 cm, bottom = 1 cm]{geometry} 

\usepackage{multicol, graphicx}

\usepackage{amsthm}
\theoremstyle{definition}
\newtheorem{problem}{Задача}
\newtheorem{definition}{Визначення}

\allowdisplaybreaks
\setlength\parindent{0pt}
\pagestyle{empty}

\newcommand{\argmax}{\arg\max}
\newcommand{\argmin}{\arg\min}

\newcommand{\dif}{\mathrm{d}}
\newcommand{\dydx}{\dfrac{\dif y}{\dif x}}
\newcommand{\dxdt}{\dfrac{\dif x}{\dif t}}
\newcommand{\dydt}{\dfrac{\dif y}{\dif t}}

\newcommand{\NN}{\mathbb{N}} 
\newcommand{\ZZ}{\mathbb{Z}}
\newcommand{\QQ}{\mathbb{Q}}
\newcommand{\RR}{\mathbb{R}}
\newcommand{\CC}{\mathbb{C}}

\newcommand{\LaReF}[1]{(\ref{#1})}

\renewcommand{\epsilon}{\varepsilon}
\renewcommand{\phi}{\varphi}

\newcommand{\ws}{\text{ }}

\newcommand{\Max}{\displaystyle\max\limits}
\newcommand{\Min}{\displaystyle\min\limits}
\newcommand{\Sum}{\displaystyle\sum\limits}
\newcommand{\Int}{\displaystyle\int\limits}
\newcommand{\Prod}{\displaystyle\prod\limits}


\begin{document}
	\setcounter{section}{1}
	
	\section{Методи розв'язання нелінійних рівнянь}
	
	\textit{Постановка задачі}. Нехай маємо рівняння $f(x)=0$, $\bar x$ -- його розв'язок, тобто $f(\bar x)=0$.\\
	
	Задача розв'язання цього рівняння розпадається на етапи:
	\begin{enumerate}
		\item Існування та кількість коренів.
		\item Відділення коренів, тобто розбиття числової вісі на інтервали, де знаходиться один корінь.
		\item Обчислення кореня із заданою точністю $\epsilon$.
	\end{enumerate}

	Для розв'язання перших двох задач використовуються методи математичного аналізу та алгебри, а також графічний метод. Далі розглядаються методи розв'язання третього епату.
	
	\subsection{Метод ділення навпіл}
	Припустимо, що на $[a,b]$ знаходиться лише один корінь рівняння \begin{equation} \label{eq:f(x)=0} f(x)=0 \end{equation} для $f(x)\in C([a,b])$ який необхідно визначити. Нехай $f(a)f(b)<0$. Припустимо, що $f(a)>0$, $f(b)<0$. Покладемо $x_1=\dfrac{a+b}{2}$ і обчислимо $f(x_1)$. Якщо $f(x_1)<0$, то шуканий корінь $\bar x$ знаходиться на інтервалі $(a,x_1)$. Якщо ж $f(x_1)>0$, то $\bar x\in(x_1,b)$. З двох інтервалів $(a,x_1)$ і $(x_1,b)$ вибираємо той, на границях якого $f(x)$ має різні	 знаки, знаходимо точку $x_2$ -- середину вибраного інтервалі, обчислюємо $f(x_2)$, і повторюємо вказаний процес.\\
	
	В результаті отримуємо послідовність інтервалів, що містять шуканий корінь $\bar x$, причому довжина кожного натсупного інтервалу вдвічі менше.\\
	
	Цей процес продовжується доки довжина $b_n-a_n$ отриманого інтервалу $(a_n,b_n)$ не стане меншою за $2\epsilon$. Тоді $x_{n+1}$, як середина інтервалу $(a_n,b_n)$, пов'язана з $\bar x$ нерівністю \begin{equation} \label{eq:xn-xbar} |x_{n+1}-\bar x|<\epsilon. \end{equation} За теоремою Больцано-Коші, ця умова буде виконуватися для деякого $n$. Справді, оскільки \[|b_{k+1}-a_{k+1}=\dfrac12|b_k-a_k|,\] то \begin{equation} \label{eq:xn-xbar 2} |x_{n+1}-\bar x|\le \dfrac1{2^{n+1}}(b-a).\end{equation} Звідси ж отримуємо нерівність для обчислення кількості ітерацій $n$ для виконання умови \LaReF{eq:xn-xbar}: \[n=n(\epsilon)\ge \left[\log\left(\dfrac{b-a}{\epsilon}\right)\right] + 1.\] Степінь збіжності лінійна, тобто геометричної прогресії зі знаменником $q=1/2$.\\
	
	Переваги методу: простота, надійність. Недоліки методу: низька швидкість збіжності, метод не узагальнюється на системи.
	
	\subsection{Метод простої ітерації}
	
	Спочатку рівняння \begin{equation} \label{eq:f(x)=0 2} f(x) = 0 \end{equation}] замінюється еквівалентним \begin{equation} \label{eq:x=phi(x)} x=\phi(x). \end{equation} 
	
	Ітераційний процес має вигляд \begin{equation} \label{eq:xn=phi(xn)} x_{n+1}=\phi(x_n), \quad n=0,1,\ldots \end{equation} 
	
	Початкове наближення $x_0$ задається. \\
	
	Для збіжності велике значення має вибір функції $\phi(x)$. Перший спосіб заміни рівняння полягає у відділенні змінної з якогось члена рівняння. Більш продуктивним є перехід від рівняння \LaReF{eq:f(x)=0 2} до \LaReF{eq:x=phi(x)} з функцією $\phi(x)=x+\tau(x)f(x)$, де $\tau(x)$ -- знакостала функція на тому відрізку, де шукаємо корінь.\\
	
	Кажуть, що ітераційний метод \textit{збігається}, якщо $\lim_{k\to\infty} x_k=\bar x$.\\
	
	Далі $U_r=\{x:|x-a|\le r\}$ відрізок довжини $2r$ з серединою в точці $a$. З'ясуємо умови, при яких збігається метод простої ітерації.\\
	
	\begin{theorem}
	\label{th:1}
	Якщо $\Max_{x\in[a,b]=U_r} |\phi'(x)|\le q<1$, то метод простої ітерації збігається і має місце оцінка \begin{equation} \label{eq:simple-iteration-convergence-bound} |x_n-\bar x|\le \dfrac{q^n}{1-q}|x_0-x_1|\le\dfrac{q^n}{1-q}(b-a) \end{equation}
	\end{theorem}

	\begin{proof} Нехай $x_{k+1},x_k\in U_r$. Тоді
	\begin{equation}
	\label{eq:2.4}
	|x_k-x_{k-1}|=|\phi(x_k)-\phi(x_{k-1})=|\phi'(\xi_k)(x_k-x_{k-1})|\le \\
	\xi_k=x_k+\theta_k(x_{k+1})-x_k), \quad 0<\theta_k<1\\
	\le |\phi'(\xi_k)|\cdot|x_k-x_{k-1}|\le q|x_k-x_{k-1}|=\cdots=q^k|x_1-x_0|\\
	|x_{k+p}-x_k|=|x_{k+p}-x_{k+p-1}+\cdots+x_{k+1}-x_k|\le |x_{k+p}-x_{k+p-1}|+\cdots+|x_{k+1}-x_k|\le\\
	\le (q^{k+p-1}+q^{k+p-2}+\cdots+q^k)|x_1-x_0|=\dfrac{q^k-q^{k+p-1}}{1-q}|x_1-x|0|\xrightarrow[k\to\infty]0
	\end{equation}
	
	Бачимо, що $\{x_k\}$ -- фундаментальна послідовність, а тому збіжна. При $p\to\infty$ в \LaReF{eq:2.4} отримуємо \LaReF{eq:simple-iteration-convergence-bound}.
	\end{proof}
	
	Визначимо кількість ітерацій для досягнення точності $\epsilon$. З оцінки в \ref{th:1} отримуємо \[x_n-\bar x|\le \dfrac{q^n}{1-q}(b-a)<\epsilon\Rightarrow n(\epsilon)=n\ge \left[\dfrac{\ln \left(\dfrac{\epsilon(1-q)}{b-a)}\right)}{\ln q}\right]+1.\]
	
	Практично ітераційний процес зупиняємо при $|x_n-x_{n-1}|<\epsilon$, але ця умов не гарантуэ $|x_n-\bar x|<\epsilon$.
	
	\begin{side_comment}
		Умова збіжності методу може бути замінена на умову Ліпщиця \[|\phi(x)-\phi(y)|\le q|x-y|, \quad 0<q<.1\]
	\end{side_comment}

	Переваги методу: простота, при $q<1/2$ метод збігається швидше ніж метод ділення навпіл, метод узагальнюєтсья на системи. Недоліки методу: при $q>1/2$ збігається повільніше ніж метод ділення навпіл, виникають труднощі при звдеені $f(x)=0$ до $\phi(x)=x$.
	
\end{document}