% cd ..\..\Users\NikitaSkybytskyi\Desktop\c3s2\numerical-analysis
% pdflatex lecture-5.tex && pdflatex lecture-5.tex && del lecture-5.out, lecture-5.log, lecture-5.aux, lecture-5.toc && start lecture-5.pdf

\documentclass[a4paper, 12pt]{article}
\usepackage[utf8]{inputenc}
\usepackage[english, ukrainian]{babel}
\usepackage{amsmath, amssymb}
\usepackage[top = 1 cm, left = 1 cm, right = 1 cm, bottom = 1 cm]{geometry} 

\usepackage{multicol, graphicx}

\usepackage{amsthm}
\theoremstyle{definition}
\newtheorem{problem}{Задача}
\newtheorem{definition}{Визначення}

\allowdisplaybreaks
\setlength\parindent{0pt}
\pagestyle{empty}

\newcommand{\argmax}{\arg\max}
\newcommand{\argmin}{\arg\min}

\newcommand{\dif}{\mathrm{d}}
\newcommand{\dydx}{\dfrac{\dif y}{\dif x}}
\newcommand{\dxdt}{\dfrac{\dif x}{\dif t}}
\newcommand{\dydt}{\dfrac{\dif y}{\dif t}}

\newcommand{\NN}{\mathbb{N}} 
\newcommand{\ZZ}{\mathbb{Z}}
\newcommand{\QQ}{\mathbb{Q}}
\newcommand{\RR}{\mathbb{R}}
\newcommand{\CC}{\mathbb{C}}

\newcommand{\LaReF}[1]{(\ref{#1})}

\renewcommand{\epsilon}{\varepsilon}
\renewcommand{\phi}{\varphi}

\newcommand{\ws}{\text{ }}

\newcommand{\Max}{\displaystyle\max\limits}
\newcommand{\Min}{\displaystyle\min\limits}
\newcommand{\Sum}{\displaystyle\sum\limits}
\newcommand{\Int}{\displaystyle\int\limits}
\newcommand{\Prod}{\displaystyle\prod\limits}


\title{Чисельні методи}
\author{Нікіта Скибицький}
\date{\today}

\begin{document}

\maketitle

\tableofcontents

\begin{theorem}[прос Чебишовський альтернанс]
	$f(x_i) - 	Q_n^0 (x_i) = \alpha (-1)^i \|f- Q_n^0\|_C$, де
	$\alpha = +\pm 1$. \\

	Точок має бути хоча б $n + 2$.
\end{theorem}

\begin{example}
	$f(x) \in C'[a, b]$, опукла. Побудувати поліном $Q(x) = c_0 + c_1 x$ найкращого рівномірного наближення.
\end{example}

\begin{solution}
	Має бути хоча б три точки Чебишовського альтернансу: $a$, $b$ і ще якась $\xi$. Давайте позначимо $L = \Delta(f) = \|f - Q_1^0\|_{C[a,b]}$. \\

	$f(a) - Q_1(a) = - L$, тому $\alpha = -1$. Маємо систему
	\begin{equation}
		\label{eq:5.3}
		\begin{aligned}
			f(a) - Q_1(a) &= -L, \\
			f(\xi) - Q_1(\xi) &= L, \\
			f(b) - Q_1(b) &= -L,
		\end{aligned}
	\end{equation}
	а також, якщо $\phi(x) = f(x) - Q_1(x)$, то $\phi'(\xi) = 0$, додається до системи \eqref{eq:5.3}. Розв'язуємо систему \eqref{eq:5.3} і знаходимо всі змінні, зокрема $c_0$, $c_1$ і $L$. \\

	\begin{remark}
		Якщо віднімемо від третього рівняння перше, то отримаємо, що $\frac{f(b) - f(a)}{b - a} = c_1$. Окрім того, $Q_1'(x) = c_1$. Тобто січна через кінці графіку паралельна поліному найкращого наближення, а також дотичній в точці $\xi$, з чого випливає геометричний підхід до побудови поліному найкращого наближення.
	\end{remark}
\end{solution}

\begin{example}
	Для $f(x) = x^{n + 1}$, $x \in [-1, 1]$, побудувати $Q_n$.
\end{example}

\begin{solution}
	Виходячи з визначення, $f(x) - Q_n(x) = \bar P_{n + 1}(x)$. Тут риска означає, що поліном $P_{n + 1}$ унітарний. Враховуючи що наша задача $\|f - Q_n\|_C \to \min$, тобто $\bar \|P_{n + 1}\|_C \to \min$, тобто $\bar P_{n + 1}$ -- унітарний поліном Чебишова степеню $n + 1$. Звідси випливає, що $f(x) - Q_n^0 (x) = \bar T_{n + 1}(x)$, тобто $Q_n^0 (x) = f(x) - \bar T_{n + 1}(x)$.
\end{solution}

\begin{example}
	Нехай $f(x) = P_{n + 1}(x)$, $x \in [a, b]$. Знайти $Q_n(x)$ НРН.
\end{example}

\begin{solution}
	Нехай $P_{n + 1}(x) = a_{n + 1} x^{n + 1} + a_n x^n + \ldots + a_0$. Тоді
	\begin{equation}
		\label{eq:5.4}
		P_{n + 1}(x) - Q^0_n(x) = a_{n + 1} \bar T_{n + 1}^{[a, b]}(x)
	\end{equation}
	Нагадаємо, що (з мінімізації похибки інтерполяційного поліному) маємо $\bar T_{n + 1}^{[a, b]} (x) = (x - x_0) \cdot (x - x_n) = \ldots = \frac{(b - a)^{n + 1}}{2^{n + 1}} \bar T_{n + 1}^{[-1, 1]} (t) = \frac{(b - a)^{n + 1}}{2^{2 n + 1}} T_{n + 1}^{[-1, 1]} (t)$. Тому з формули \eqref{Peq:5.4} маємо 
	\begin{equation}
		\label{eq:5.5}
		P_{n + 1}(x) - Q_n^0(x) = a_{n + 1} \cdot \frac{(b - a)^{n + 1}}{2^{2n + 1}} \cdot T_{n + 1}^{[a, b]} \left( \frac{2x - b - a}{b - a}\right)
	\end{equation}
\end{solution}

Розгялнемо ситуацію у якій ми не можемо побудувати ЕНН, але хочемо розв'язати задачу $\|f - Q_n(x)\|\le \epsilon$, і нехай  $f \in C^{n + 1}[a, b]$, і хочемо невелике $n$.

\begin{example}
	Нехай задача $f(x)$, де $x \in [-1, 1]$. Побудувати поліном якомога меншого степеню який би наближав функцію  $f$ з точністю  $\epsilon$ в нормі $C$.
\end{example}

\begin{solution}
	Запишемо розвинення в ряд, причому із ``запасом'': $f(x) = P_n(x) - r_n(x)$, де $P_n(x) = \sum_{k = 0}^n a_k x6k = a_0 + a_1 x + \ldots + a_n x^n$, причому $n$ беремо таким, щоб $\|f - P_n\| \le \frac\epsilon2$. Наблизимо тепер на проміжку $[-1, 1]$ поліном $P_n(x)$ поліномом $P_{n - 1}(x)$ як НРН. Тоді 
	\begin{equation}
		\label{eq:5.6}
		P_n - P_{n - 1} = a_n \cdot \bar T_n (x) = \frac{a_n}{2^{n - 1}} \cdot T_n(x),
	\end{equation}
	тоді
	\begin{equation}
		\label{eq:5.7}
		\|P_n - P_{n - 1}\|_C \le \frac{a_n}{2^{n - 1}},
	\end{equation}
	а тоді 
	\begin{equation}
		\label{eq:5.8}
		P_{n - 1}(x) = P_n(x) - \frac{a_n}{2^{n - 1}} \cdot T_n(x),
	\end{equation}
	а тому
	$\|f - P_{n - 1}\| \le \|f - P_n + P_n - P_{n - 1} \| \le \|f - P_n\| + \|P_n - P_{n - 1}\| \le \frac\epsilon2 + \frac{a_n}{2^{n - 1}}$. \\

	Якщо останнє число менше за $\esilon$, то можемо $P_n$ замінити на $P_{n - 1}$.
\end{solution}

...

\section{Чисельне інтегрування}

\begin{equation}
	\label{eq:6.1}
	I = I(f) = \int_a^b \rho(x) f(x) \diff x,
\end{equation}
де $\rho > 0$. \\

Введемо сітку $\omega_h = \{ x_i = a + i h \}$, де $h = \frac{b - a}{N}$, $i = \overline{0, N}$. Можемо побудувати наступну квадратурну суму:
\begin{equation}
	\label{eq:6.2}
	I_n (f) = \sum_{k = 0}^n f(\xi_k) \Delta x_k,
\end{equation}
де $\Delta_k = x_k - x_{k - 1}$. \\

Як бачимо $I_n(f) \xrightarrow[\Delta x_k \to 0]{} I(f)$. У згальному випадку ми тепер можемо записати наступним чином: 
\begin{equation}
	\label{eq:6.3}
	I_n(f) = \sum_{k = 0}^m c_k \cdot f(x_k),
\end{equation}
яка називається \textit{квадратнурною сумою} або \textit{квадратурною формою}. А квадратурна формула має вигляд
\begin{equation}
	\label{eq:6.4}
	I(f) \approx I_n(f).
\end{equation}

\begin{definition}
	Величина 
	\begin{equation}
		\label{eq:6.5}
		R_n(f) = I(f) - I_n(f)
	\end{equation}
	називається \textit{похибкою квадратурної формули}.
\end{definition}

\begin{definition}
	Будемо говорити, що квадратична формула \textit{має $m$-ий алгебраїчний степінь точності} якщо вона є точною на алгебраїчних поліномах степеню до $m$ включно:
	\begin{equation}
		\label{eq:6.6}
		R_n(P_m(x)) = 0,
	\end{equation}
	але не на степені $m + 1$:
	\begin{equation}
		\label{eq:6.7}
		\exists P_{m + 1}: R_n(P_{m + 1}(x)) \ne 0.
	\end{equation}
\end{definition}

Зрозуміло, що можна перевіряти еквівалентні умови:
\begin{equation}
	\label{eq:6.6_prime}
	R_n(x^\alpha) = 0, \quad \alpha = \overline{0, m}
\end{equation}
\begin{equation}
	\label{eq:6.7_prime}
	R_n(x^{m + 1}) \ne 0.
\end{equation}

\subsection{Еквівалентні квадратурні формули}

Зрозуміло, що можемо записати:
\begin{equation}
	\label{eq:6.8}
	I = \int_a^b \rho(x) f(x) \diff x = \sum_{i = 1}^N \int_{x_{i-1}}^{x_i} \rho(x) f(x) \diff x
\end{equation}

\subsection{Інтерполяційний підхід}

Подамо $f(x)$ у вигляді f(x) = L_n(x) + r_n(x)$, тоді будемо мати, що 
\begin{equation}
	\label{eq:6.9}
	I(f) = I(L_n + r_n) = I(L_n) + I(r_n) = I_n(f) + R_n(f),
\end{equation}
зрозуміло, що 
\begin{align}
	\label{eq:6.10}
	I(L_n) &= I_n(f), \\
	\label{eq:6.11}
	I(r_n) &= R_n(f).
\end{align}

Причому, враховуючи $r_n(x) = \frac{f^{(n + 1)}(\xi)}{(n + 1)!} \cdot \omega_n(x)$, ортримуємо
\begin{equation}
	\label{eq:6.12}
	R_n(x) = \int_a^b \frac{f^{(n + 1)}(\xi)}{(n + 1)!} \cdot \omega_n(x) \diff x
\end{equation}

Це так звана \textit{головна формула}. У свою чергу є \textit{складена формула}:
\begin{equation}
	\label{eq:6.14}
	I_h(f) = \sum_{1 = 1}^N \int_{x_{i - 1}}^{x_i} \rho(x) f(x) \diff x = \sum_{i = 1}^N I_{n, i}(f).
\end{equation}

Перейдемо до розгляду найпростіших головних та складених квадратурних форм. Для геометричних інтерпретацій покладемо $\rho \equiv 1$.

\subsection{Формули прямокутників}

\textbf{Формула лівих прямокутників} ($n = 0$): $L_0(x) = f(x_{i - 1})$, 
\begin{align}
	\label{eq:6.15}
	I_i(f) &= \int_{x_{i-1}}^{x_i} f(x) \diff x, \\
	\label{eq:6.16}
	I_{h, i}^L (f) &= h \cdot f(x_{i - 1}), \\
	\label{eq:6.17}
	R_{h, i}^L (F) &= \int_{x_{i - 1}}^{x_i} f'(\xi_i) (x - x_{i - 1}) \diff x \approx \frac{h^2 \cdot f'(\eta_i)}{2}.
\end{align}

Алгебраїчний степінь точності цієї формули -- 0. Справді
$I_i(1) = x_i - x_{i - 1} = h = I_{h, i} (1)$, але $I_i(x) = \frac{x_i^2 - x_{i - 1}^2}{2} = \frac{h(x_i + x_{i - 1})}{2} \ne h x_{i - 1} = I_{i, h}(x)$. \\

Алгебраїчну степінь точності можна ще визначити з формули \eqref{eq:6.17}.

\begin{problem}
	Формула правих прямокутників.
\end{problem}

\begin{solution}

\end{solution}

\end{document}
