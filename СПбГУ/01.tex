\section{Наближене розв'язування нелінійних рівнянь і систем}

\subsection{Наближене розв'язування нелінійних рівнянь}

Задано рівняння
\begin{equation}
	\label{eq:1}
	f(x) = 0,
\end{equation}
де функція $f(x)$ визначена і неперервна на деякому скінченному чи нескінченному інтервалі $a < x < b$.  Всяке значення $x^*$ таке, що $f(x^*) = 0$ називається коренем рівняння (\ref{eq:1}) або нулем функції $f(x)$. \\

Наближене знаходження ізольованих дійсних коренів рівняння складається з двох етапів:
\begin{enumerate}
	\item Відділення коренів, тобто встановлення проміжків $[\alpha, \beta]$, у яких міститься один і тільки один корінь рівняння (\ref{eq:1}). Використання цих проміжків для визначення початкових наближень до коренів.
	\item Уточнення наближених коренів.
\end{enumerate}

\subsubsection{Відділення коренів}

Для відділення коренів потрібно побудувати таблицю значень функції або графік функції, знайти проміжки, на кінцях яких функція $f(x)$ має різні знаки. Тоді всередині цих проміжків міститься принаймні один корінь рівняння $f(x) = 0$. Необхідно тим чи іншим способом упевнитися, що цей корінь є єдиним. \\

Для зменшення довжин проміжків може бути використаний метод ділення навпіл (бісекції). Покладаємо $[a_0, b_0] = [a, b]$. Нехай $c_0 = (a_0 + b_0) / 2$. Далі будуємо послідовність проміжків $\{ [a_k, b_k] \}$, $k = 1, 2, \ldots$ 
\[ [a_k, b_k] = \begin{cases} [a_{k-1}, c_{k-1}], & \text{якщо } f(a_{k-1}) \cdot f(c_{k-1}) < 0, \\ [c_{k-1}, b_{k-1}], & \text{якщо } f(c_{k-1}) \cdot f(b_{k-1}) < 0. \end{cases}\]

Тут на кожному кроці довжина проміжку зменшується удвічі, так що
\[ b_k - a_k = (b - a) / 2^k. \]

\subsubsection{Уточнення коренів}

Для уточнення коренів використовують ітераційні методи. При розв'язування задачі ітераційними методами варто звертати увагу на наступні моменти:
\begin{itemize}
	\item розрахункова формула;
	\item умова збіжності;
	\item порядок збіжності (швидкість збіжності): $\alpha \ge 1$ називається порядком збіжності послідовності $\{ x_k \}$ до $x^*$, якщо $x_k \to x^*$ і існує стала $C$, така що для всіх $k$: $|x_k - x^*| \le C \cdot |x_{k-1} - x^*|^\alpha$;
	\item критерій отримання розв'язку із заданою точністю $\epsilon$. Тут мається на увазі умова на $x_k$,  яка забезпечує $|x_k - x^*| < \epsilon$.
\end{itemize}

Існують оцінки для фактичної похибки $|x_k - x^*|$ апріорні та апостеріорні. Апріорні оцінки часто бувають сильно завищені. Легко показати, що для оцінки точності наближення $x_k$ довільного ітераційного методу можна скористатися нерівністю 
\begin{equation}
	\label{eq:2}
	|x_k - x^*| \le \frac{|f(x_k)|}{m_1},
\end{equation}
де $m_1 = \min |f'(x)|$ при $a \le x \le b$.

\subsubsection{Метод Ньютона (метод дотичних)}

Нехай задано рівняння (\ref{eq:1}). Припускаємо, що функція $f(x)$ -- дійсна і шукаємо дійсний корінь $x^*$. Будемо припускати, що на відрізку $[a, b]$, такому що $f(a) \cdot f(b) < 0$, існують неперервні похідні $f'(x) \ne 0$, $f''(x) \ne 0$. Вибираємо $x_0 \in [a, b]$. Замінимо рівніння в околі $x_0$ наближено рівнянням
\[ f(x_0) + f'(x_0) \cdot (x - x_0) = 0, \]
ліва частина якого є лінійною частиною розкладу функції $f(x)$ у ряд Тейлора в околі точки $x_0$. Звідси
\[ x_1 = x_0 - \frac{f(x_0)}{f'(x_0)}. \]

Діючи аналогічно, отримуємо розрахункову формулу методу Ньютона
\begin{equation}
	\label{eq:3}
	x_k = x_{k-1} - \frac{f(x_{k-1})}{f'(x_{k-1})}, \quad k = 1, 2, \ldots.
\end{equation}

Метод Ньютона має простий геометричний зміст: $x_k$ є абцисою точки перетину дотичної до графіку $f(x)$, побудованій у точці $(x_{k-1}, f(x_{k-1}))$, з віссю абцис.

\begin{theorem}[про збіжність]
	\label{th:1}
	Якщо
	\begin{enumerate}
		\item $f(a) \cdot f(b) < 0$,
		\item $f'(x)$, $f''(x)$ відмінні від нуля (зберігають знаки при $x \in [a, b]$),
		\item $f(x_0) \cdot f''(x_0) > 0$, $x_0 \in [a, b]$,
	\end{enumerate}
	то $x_k \to x^*$, причому швидкість збіжності визначається нерівністю
	\[ |x_k - x^*| \le \frac{M_2}{2m_1} \cdot (x_{k-1} - x^*)^2. \]
	Тут $m_1 = \Min_{x \in [a, b]} |f'(x)|$, $M_2 = \Max_{x \in [a, b]} |f''(x)|$.
\end{theorem}

Якщо $f(x_0) \cdot f''(x_0) < 0$, то можна не прийти до $x = x^*$, якщо $x_0$ не дуже гарне. \\

Оскільки метод Ньютона має другий порядок збіжності, то можна користуватися наступним критерієм: якщо $|x_k - x_{k-1}| < \epsilon$, то $|x_k - x^*| < \epsilon$.
\begin{remark}
	Якщо $f'(x^*) = 0$, то квадратичної збіжності може і не бути. \\

	Наприклад, нехай $f(x) = x^2$, корінь $x^*$ -- корінь другої кратності, розрахункова формула має вигляд $x_{k+1} = x_k / 2$ і збіжність лінійна. \\

	Другого порядку збіжності для кореня кратності $p$ можна досягнути, застосовуючи розрахунков формулу вигляду
	\begin{equation}
		\label{eq:4}
		x_k = x_{k-1} - p \cdot \frac{f(x_{k-1})}{f'(x_{k-1})}, \quad k = 1, 2, \ldots .
	\end{equation}
\end{remark}

\begin{remark}
	Інколи має сенс застосовувати модифікований метод Ньютона з розрахунковою формулою
	\begin{equation}
		\label{eq:5}
		x_k = x_{k-1} - \frac{f(x_{k-1})}{f'(x_0)}, \quad k = 1, 2, \ldots .
	\end{equation}
	Швидкість збіжності модифікованого методу значно менша.
\end{remark}

\begin{remark}
	Наведемо розрахункову формулу методу 3-го порядку збіжності
	\begin{equation}
		\label{eq:6}
		x_k = x_{k-1} - \frac{f(x_{k-1})}{f'(x_{k-1})} - \frac{f^2(x_{k-1} \cdot f''(x_{k-1}))}{2 (f'(x_{k-1}))^3}, \quad k = 1, 2, \ldots .
	\end{equation}
\end{remark}

\subsubsection{Метод січних}

Замінюючи похідну у розрахунковій формулі методу Ньютона її наближеним значенням за формулами чисельного диференціювання , отримуємо розрахункову формулу
\begin{equation}
	\label{eq:7}
	x_{k+1} = x_k - \frac{f(x_k)}{f(x_k) - f(x_{k-1})} \cdot (x_k - x_{k-1}), \quad k = 1, 2, \ldots.
\end{equation}

Порядок збіжності методу січних визначається нерівністю
\[ |x_k - x^*| \le \frac{M_2}{2m_1} \cdot (x_{k-1} - x^*)^\alpha, \]
де $\alpha = \frac{\sqrt{5} + 1}{2} \approx 1.618$. \\

У методі січних можна користатися критерієм: якщо $|x_k - x_{k-1}| < \epsilon$, то $|x_k - x^*| < \epsilon$.

\subsubsection{Метод хорд}

Нехай відомий проміжок $[a, b]$ такий, що $f(a) \cdot f(b) < 0$ і $f''(x) > 0$. Розглянемо два можливих випадки.
\begin{enumerate}
	\item $f(a) < 0$, відповідно $f(b) > 0$. У цьому випадку кінець $b$ нерухомий і послідовні наближення при $x_0 = a$
	\begin{equation}
	 	\label{eq:8}
	 	x_{k+1} = x_k - \frac{f(x_k)}{f(b) - f(x_k)} \cdot (b - x_k), \quad k = 0, 1, \ldots .
	\end{equation} 
	утворюють монотонно зростаючу послідовність, причому 
	\[ a = x_0 < x_1 < x_2 < \ldots < x_k < x_{k+1} < \ldots < x^* < b. \]
	
	\item $f(a) > 0$, відповідно $f(b) < 0$. У цьому випадку кінець $a$ нерухомий і послідовні наближення при $x_0 = b$
	\begin{equation}
	 	\label{eq:8}
	 	x_{k+1} = x_k - \frac{f(x_k)}{f(x_k) - f(a)} \cdot (x_k - a), \quad k = 0, 1, \ldots .
	\end{equation}
	утворюють монотонно спадну послідовність, причому 
	\[ b = x_0 > x_1 > x_2 > \ldots > x_k > x_{k+1} > \ldots > x^* > a. \]
\end{enumerate}

Границі цих послідовностій $x^*$ існують, оскільки вони (послідовності) обмежені і монотонні. \\

Для оцінки точності можна скористатися уже відомою нерівністю (\ref{eq:2})
\[ |x_k - x^*| \le \frac{|f(x_k)|}{m_1}, \]
і
\begin{equation}
	\label{eq:10}
	|x_k - x^*| \le \frac{M_1 - m_1}{m_1} \cdot |x_k - x_{k-1}|,
\end{equation}

де $m_1 = \min |f'(x)$, $M_1 = \max |f'(x)|$ при $a \le x \le b$. \\

Геометрично метод еквівалентній заміні кривої $y = f(x)$ хордами, що проходять через точки $(x_k, f(x_k))$, $(x_{k-1},f(x_{k-1}))$, $k = 0, 1, \ldots$. \\

Порядок методу -- перший і не можна користуватися у якості критерію модулем різниці двох послідовних наближень.

\subsubsection{Метод простої ітерації (метод послідовних наближень)}

Замінимо рівняння (\ref{eq:1}) рівносильним рівнянням
\begin{equation}
	\label{eq:11}
	x = \phi(x),
\end{equation}
де $\phi(x)$ -- неперервна. Розрахункова формула методу
\begin{equation}
	\label{eq:12}
	x_k = \phi(x_{k-1}), \quad k = 1, 2, \ldots .
\end{equation}
якщо ця послідовність збіжна, то $\Lim_{k \to \infty} x_k = x^*$. \\

Коротко сформулюємо умову збіжності. Нехай у деякому околі $(a, b)$ кореня $x^*$ рівняння (\ref{eq:11}) похідна $\phi'(x)$ зберігає знак і виконується нерівність
\begin{equation}
	\label{eq:13}
	|\phi'(x)| \le q < 1.
\end{equation}
Тоді, якщо похідна $\phi'(x)$ додатня, то послідовні наближення (\ref{eq:12}) сходять до кореня $x^*$ монотонно, а якщо похідна $\phi'(x)$ від'ємна, то послідовні наближення коливаються навколо $x^*$. \\

Апріорна оцінка похибки
\begin{equation}
	\label{eq:14}
	|x^* - x_k| \le \frac{q^k}{1 - q} \cdot |x_1 - x_0|. 
\end{equation}

Апостеріорна оцінка похибки
\begin{equation}
	\label{eq:15}
	|x^* - x_k| \le \frac{q}{1 - q} \cdot |x_k - x_{k-1}|. 
\end{equation}

Відзначимо, що як показує оцінка (\ref{eq:15}), помилково було би використовувати у якості критерію отримання розв'язку із заданою точністю $\epsilon$ рівності $x_k$ і $x_{k-1}$ з точністю $\epsilon$.
\begin{remark}
	Нагадаємо, що приводити рівняння вигляду (\ref{eq:1}) до вигляду (\ref{eq:11}) варто так, аби $|\phi'(x)| \le q < 1$, причому, чим менше число $q$, тим швидше, взагалі кажучи, послідовні наближення зійдуться до кореня $x^*$. Вкажемо один достатньо загальний прийом зведення. Нехай шуканий корінь $x^*$  рівняння лежить на відрізку $[a, b]$, причому
	\[ 0 < m_1 \le f'(x) \le M_1 \]
	при $a \le x \le b$. Замінимо рівняння (\ref{eq:1}) еквівалентним йому рівнянням
	\[ x = x - \lambda \cdot f(x) \quad (\lambda > 0). \]
	З умови збіжності отримуємо, що можна взяти 
	\[ \lambda = \frac{1}{M_1} \]
	і тоді 
	\[ q = 1 - \frac{m_1}{M_1} < 1 \]
\end{remark}

\begin{remark}
	Формулу (\ref{eq:3}) методу Ньютона можна розглядати як формулу методу ітерацій для рівняння $x = \phi(x)$, де $\phi(x) = f(x) / f'(x)$.
\end{remark}

Легко перевірити, що $\phi'(x^*) = 0$. Тому варто очікувати квадратичну збіжність методу.

\subsubsection{Завдання}

Задано рівняння $f(x)$. Вимагається
\begin{enumerate}
	\item Відділити усі корені, або корені на вказаному інтервалі. 
	\item Звузити інтервали, визначені вище, у декілька разів, використовуючи метод ділення навпіл.
	\item Обчислити корені методом Ньютона (або модифікованим) з точністю $\epsilon = 10^{-6}$. Ці значення коренів надалі будемо вважати ``точними'', раніше і пізніше в таблиці вони позначені $x^*$.
	\item Використовуючи інтервали з 1-го чи 2-го пункту, знайти потрібні корені з точністю $\epsilon = 10^{-4}$ методом січних. У якості критерію використовувати модуль різниці між двума сусідніми наближеннями. Порівняти з фактичною похибкою.
	\item Використовуючи інтервали з 1-го чи 2-го пункту, знайти потрібні корені з точністю $\epsilon = 10^{-3}$ методом хорд. У якості критерію використовувати оцінку (\ref{eq:10}). Порівняти з фактичною похибкою.
	\item Обчислити корені методом ітерацій з точністю $\epsilon = 10^{-5}$, вибравши у якості $x_0$ те ж значення, що і в методі Ньютона.
	\item Порівняти результати, кількість ітерацій.
\end{enumerate}

Принаймні для методу Ньютона повинна бути створені підпрограма з параметрами
\begin{itemize}
	\item $x_0$ -- нульове наближення до кореня;
	\item $\epsilon$ -- задана точність;
	\item $\text{kmax}$ -- максимальна кількість ітерацій (для виключення зациклювання).
\end{itemize}

Підпрограма повинна повертати або $x_k$ таке, що $|x_k - x_{k - 1}| < \epsilon$, або $x_{\text{kmax}}$. Результати методів оформити у вигляді таблиці:
\begin{table}[H]
	\centering
	\begin{tabular}{|c|c|c|c|c|}
		\hline
		$k$ & $x_k$ & $x_k - x_{k-1}$ & $x_k - x^*$ & $f(x_k)$ \\ \hline
		0 & & -- & & \\ \hline
		1 & & & & \\ \hline
		$\vdots$ & $\vdots$ & $\vdots$ & $\vdots$ & $\vdots$ \\ \hline
	\end{tabular}
\end{table}

\subsection{Метод Ньютона для розв'язування системи 2-х рівнянь}

Задана система 
\begin{equation}
	\label{eq:16}
	\left\{
		\begin{aligned}
			& f(x, y) = 0, \\
			& g(x, y) = 0,
		\end{aligned}
	\right.
\end{equation}
де $f(x, y)$, $g(x, y)$ достатньо гладкі функції. \\

У результаті дій, аналогічних випадку одного рівняння, тобто приблизно замінюючи систему (\ref{eq:16}) лінійною системо, отримуємо наступні розрахункові формули:
\[ x_{k+1} = x_k - \frac{d_x^{(k)}}{d^{(k)}}, \quad y_{k+1} = y_k - \frac{d_y^{(k)}}{d^{(k)}}, \]
де
\[ d^{(k)} = \begin{vmatrix} f_x'(x_k, y_k) & f_y'(x_k, y_k) \\ g_x'(x_k, y_k) & g_y'(x_k, y_k) \end{vmatrix}, \]
\[ d_x^{(k)} = \begin{vmatrix} f(x_k, y_k) & f_y'(x_k, y_k) \\ g(x_k, y_k) & g_y'(x_k, y_k) \end{vmatrix}, \quad d_y^{(k)} = \begin{vmatrix} f_x'(x_k, y_k) & f(x_k, y_k) \\ g_x'(x_k, y_k) & g(x_k, y_k) \end{vmatrix}. \]

\subsubsection{Завдання}

Вимагається знайти всі розв'язки системи рівнянь, або розв'язки в заданій області із заданою точністю $\epsilon = 10^{-6}$. Програма має містити підпрограму для уточнення розв'язку методом ньютона з параметрами:
\begin{itemize}
	\item $x_0$, $y_0$ -- нульове наближення до кореня;
	\item $\epsilon$ -- задана точність;
	\item $\text{kmax}$ -- максимальна кількість ітерацій (для виключення зациклювання).
\end{itemize}

Підпрограма повинна повертати або $(x_k, y_k)$ таке, що 
\[ \| (x_k - x_{k - 1}, y_k - y_{k-1})\| < \epsilon,\] 
або $(x_{\text{kmax}}, y_{\text{kmax}})$. Тут норма вектора $(x_k - x_{k - 1}, y_k - y_{k-1})$ може бути обчислена, наприклад, наступним чином: 
\[ \| (x_k - x_{k - 1}, y_k - y_{k-1})\| = \sqrt{(x_k - x_{k - 1})^2 + (y_k - y_{k-1})^2 }.\]
Звіт повинен містити:
\begin{enumerate}
	\item Графіки функцій $f(x,y)=0$ і $g(x,y)=0$ для вибору початкового наближення.
	\item Уточнення початкового наближення до того часу, коли $\| (x_k - x_{k - 1}, y_k - y_{k-1})\| < \epsilon$, методом Ньютона.
\end{enumerate}
Результати оформити у вигляді таблиці:
\begin{table}[H]
	\centering
	\begin{tabular}{|c|c|c|c|c|c|}
		\hline
		$k$ & $x_k$ & $y_k$ & $\| (x_k - x_{k - 1}, y_k - y_{k-1})\|$ & $f(x_k, y_k)$ & $g(x_k, y_k)$ \\ \hline
		0 & & & -- & & \\ \hline
		1 & & & & & \\ \hline
		$\vdots$ & $\vdots$ & $\vdots$ & $\vdots$ & $\vdots$ & $\vdots$ \\ \hline
	\end{tabular}
\end{table}