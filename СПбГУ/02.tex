\section{Інтерполювання за значеннями функції. Інтерполюючий поліном у формі Ньютона. Інтерполюючий поліном у формі Лагранжа}

\subsection{Постановка задачі інтерполювання}

Нехай на проміжку $[a, b]$ задана таблиця значень дійсної функції $y = f(x)$:
\begin{table}[H]
	\centering
	\begin{tabular}{|c|c|}
		\hline
		$x$ & $f(x)$ \\ \hline
		$x_0$ & $f(x_0)$ \\ \hline
		$x_1$ & $f(x_1)$ \\ \hline
		$x_2$ & $f(x_2)$ \\ \hline
		$\vdots$ & $\vdots$ \\ \hline
		$x_n$ & $f(x_n)$ \\ \hline
	\end{tabular}
\end{table}

Вузли вважаються попарно різними: $x_i \ne x_j$ при $i \ne j$. \\

Вимагається знайти значення функції в точці $x = \overline{x}$ відмінній від вузлів. \\

Наближене значення функції $f(\overline{x})$ може бути знайденим як значенні інтерполюючого полінома: $f(\overline{x}) \approx P_n(\overline{x})$, де $P_n(x)$ будується єдиним чином з умов $P_n(x_i) = f(x_i)$, $i= 0,1,\ldots,n$. \\

Похибка інтерполювання знаходиться з теореми:

\begin{theorem}
	\label{th:2.1}
	Нехай функція $f(x)$ має скінченну неперервну похідну $f^{(n+1)}(x)$ на найменшому проміжку $[c, d]$ що містить вузли інтерполювання $x_0, x_1, \ldots, x_n$ і точку $\overline{x}$, тобто $c = \min\{x_0, x_1, \ldots, x_n, \overline{x}\}$, $d = \max\{x_0, x_1, \ldots, x_n, \overline{x}\}$. \\

	Тоді  існує така точка $\xi = \xi(\overline{x})$, $c < \xi < d$, що
	\begin{equation}
		\label{eq:2.1}
		R_n(f,\overline{x}) = f(\overline{x}) - P_n(\overline{x}) = \frac{f^{(n+1)}(\xi)}{(n+1)!} \cdot \omega_{n+1}(\overline{x}), \quad \omega_{n+1}(x) = \Prod_{i=0}^n (x - x_i).
	\end{equation}

	Оцінка похибки інтерполювання обчислюється наступним чином:
	\begin{equation}
		\label{eq:2.2}
		|R_n(\overline{x}) \le M_{n+1} \cdot \frac{|\omega_{n+1}(\overline{x})|}{(n+1)!},
	\end{equation}
	де
	\begin{equation*}
		M_{n+1} = \max |f^{(n+1)}(x)|, \quad x\in[c,d].
	\end{equation*}
\end{theorem}

Часто на практиці будується поліном $P_m(x)$, де $m < n$, по $m + 1$ вузлу. Очевидно, що з $n + 1$ вузла варто вибирати такі $m + 1$, які мінімізують похибку, тобто вузли, найближчі до точки $\overline{x}$. \\

При побудові інтерполюючого полінома у вигляді $P_n(x) = a_0 x^n + a_1 x^{n-1} + \ldots + a_n$ коефіцієнти $a_i$ є розв'язком системи $P_n(x_i) = f(x_i)$, $i = 0, 1, \ldots, n$. Визначник цієї системи -- визначник Вандермонда. Він відмінний від нуля аджу вузли попарно різні. \\

Зручніше будувати поліном у формі Ньютона або у формі Лагранжа.

\subsection{Інтерполюючий поліном у формі Ньютона. Розділені різниці}

Інтерполюючий поілном у формі Ньютона має вигляд
\begin{equation}
	\label{eq:2.3}
	P_n(x) = A_0 + A_1 (x - x_0) + A_2 (x - x_0) (x - x_1) + \ldots + A_n (x - x_0) \cdots (x - x_{n-1}).
\end{equation}

Перевагою такої форми є простота знаходження коефіцієнтів: $A_0 = f(x_0)$, $A_1 = \frac{f(x_1) - f(x_0)}{x_1 - x_0}$ і так далі, а також той факт, що
\begin{equation*}
	P_K(x) = P_{k-1}(x) + A_k(x - x_0) \cdots (x - x_{k - 1}).
\end{equation*}

Якщо вузли інтеполювання $x_0, x_1, \ldots, x_n$ вибрані у порядку віддалення від точки $\overline{x}$, то можна стверджувати, що багаточлен довільного степеню $P_0(x), P_1(x), \ldots, P_n(x)$ забезпечує мінімальну похибку $f(\overline{x}) - P_n(\overline{x})$ серед усіх багаточленів заданого степеню, побудованих за заданою таблицею вузлів.\\

Розділені різниці обчислюються за формулами:
\begin{align}
	\label{eq:2.4}
	\text{р.р. 1-го пор.} & f(x_i, x_{i+1}) = \frac{f(x_{i+1}) - f(x_i)}{x_{i+1}-x_i}, \\
	\text{р.р. 2-го пор.} & f(x_i, x_{i+1}, x_{i+2}) = \frac{f(x_{i+1},x_{i+2}) - f(x_i,x_{i+1})}{x_{i+2}-x_i}, \\
	\text{р.р. }n\text{-го пор.} & f(x_0,x_1, \ldots, x_n) = \frac{f(x_1,\ldots,x_n) - f(x_0,\ldots,x_{n-1})}{x_n-x_0}. 
\end{align}

Можна довести, що 
\begin{equation*}
	f(x_0, x_1, \ldots, x_n) = \frac{f^{(n)}(\xi)}{n!},\quad \xi \in(c,d).
\end{equation*}

Можна показати, що коефіцієнти $A_i$ в інтерполюючому поліномі у формі Ньютона є розділеними різницями $i$-го порядку: $A_i = f(x_0, x_1, \ldots, x_i)$. \\

Відзначимо, що якщо метою побудови інтерполюючого поліному є не мінімізація похибки у точці інтерполювання, а мінімізація похибки на всьому інтервалів $[c, d]$, то в ролі вузлів необхідно брати корені поліному Чебишова першого роду зведеного на проміжок $[c, d]$.

\subsection{Завдання}

Задана функція $y = f(x)$, вузли. \\

Необхідно побудувати аналітичний вираз інтерполюючого поліному для функції $f(x)$ у формі Ньютона 0-го, 1-го, 2-го, 3-го степеню за заданими вузлами. Обчислити наближене значення функції у заданій точці, фактичну похибку, оцінити теоретичну.

\subsection{Інтерполюючий поліном у формі Лагранжа}

Інтерполюючий поліном у формі Лагранжа має вигляд
\begin{equation}
	\label{eq:2.5}
	P_n(x) = \Sum_{k=0}^n \frac{\omega_{n+1}(x)}{(x - x_k) \cdot \omega_{n+1}'(x_k)} = \Sum_{k=0}^n \frac{\prod_{i\ne k}(x - x_i)}{(x - x_k) \cdot \prod_{i\ne k}(x_k - x_i)} \cdot f(x_k).
\end{equation}

\subsection{Завдання}

\begin{enumerate}
	\item Задана функція $y = f(x)$, вузли, значення функції $\overline{y}$. Побудувати таблицю значень функції у вузлах. \\

	Вимагається наближено знайти таке $\overline{x}$, що $f(\overline{x}) = \overline{y}$ трьома способами:
	\begin{enumerate}
		\item ``точно'' використовуючи аналітичний вигляд оберненої функції. Позначимо $x^*$.
		\item апроксимацією функції $f(x)$ інтерполюючим поліномом $P_n(x)$ ($n \ge 2$) у форму Лагранжа і наближеним розв'язком рівняння $P_n(x) = \overline{y}$  методом ітерацій або методом січних. Позначимо розв'язок рівняння $P_n(x) = \overline{y}$ через $x_{uer}$.
		\item якщо існує однозначна обернена функція $f^{-1}(y)$, то поміняти ролями вузли і значення функції і наближено замінити обернену функцію інтерполюючим поліномом $Q_m(y)$ ($m=0,1,2,\ldots$) у формі Лагранжа і обчислити $x_m = Q_m(\overline{y})$.
		\item 
	\end{enumerate}
	Результати навести у таблицях вигляду
	\begin{table}[H]
		\centering
		\begin{tabular}{|c|c|c|c|}
			\hline
			$m$ & $x_m$ & $x_m - x_{m-1}$ & $x_m - x^*$ \\ \hline
			0 & & -- & \\ \hline
			1 & & & \\ \hline
			2 & & & \\ \hline
			$\vdots$ & $\vdots$ & $\vdots$ & $\vdots$ \\ \hline
		\end{tabular}
	\end{table}
	\item Задана функція $y = f(x)$, $[a,b] = [-1,1]$. \\

	Вимагається побудувати при різних $n$ інтерполюючі поліноми $P_n(x)$ у формі Лагранда за рівновіддаленими вузлами і за вузлами поліному Чебишова. Порівняии на графіку з функцією в одних координатних вісях.
\end{enumerate}