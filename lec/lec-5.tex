\section{Проблема власних значень}

Нехай задано матрицю $A$ розмірності $n \times n$. Тоді задача на власні значення ставиться так: знайти число $\lambda$ та вектор $\vec x \ne \vec 0$, що задовольняють рівнянню
\begin{equation}
	\label{eq:5.1}
	A \vec x = \lambda \vec x.
\end{equation}

$\lambda$ називається власним значенням $A$, а $\vec x$ -- власним вектором. З (\ref{eq:5.1}) \[ \det (A - \lambda E) \equiv P_n(\lambda) \equiv (-1)^n \lambda^n + a_n \lambda^{n-1} + \ldots + a_0 = 0. \]

Тут $P_n(\lambda)$ -- характеристичний багаточлен. \\

Для розв’язання багатьох задач механіки, фізики, хімії потрібне знаходження всіх власних значень $\lambda_i$, $i=\overline{1,n}$, а іноді й всіх власних векторів $\vec x_i$, що відповідають $\lambda_i$. Цю задачу називають повною проблемою власних значень. \\

В багатьох випадках потрібно знайти лише максимальне або мінімальне за модулем власне значення матриці. При дослідженні стійкості коливальних процесів іноді потрібно знайти два максимальних за модулем власних значення матриці. \\

Останні дві задачі називають \textit{частковими проблемами власних значень}.

\subsection{Степеневий метод}
\begin{enumerate}
	\item \textit{Знаходження} $\lambda_{\max}: |\lambda_1|\equiv\lambda_{\max}> |\lambda_2| \ge |\lambda_3| \ge \ldots$ \\

	Нехай $\vec x^{(0)}$ -- заданий вектор, будемо послідовно обчислювати вектори
	\begin{equation}
		\label{eq:5.2}
		\vec x^{(k+1)} = A \vec x^{(k)}, \quad k = 0, 1, \ldots
	\end{equation}

	Тоді $\vec x^{(k)} = A^k \vec x^{(0)}$. Нехай $\{\vec e_i\}_{i=1}^n$ -- система власних векторів. Представимо $\vec x^{(0)}$ у вигляді: \[ \vec x^{(0)} = \Sum_{i=1}^n c_i \vec e_i. \]

	Оскільки $A \vec e_i = \lambda_i \vec e_i$, то $\vec x^{(k)} = \Sum_{i=1}^n c_i \lambda_i^k \vec e_i$. При великих $k$: $\vec x^{(k)} \approx c_1 \lambda_1^k \vec e_1$. \\

	Тому \[ \mu_1^{(k)} = \dfrac{\vec x_m^{(k+1)}}{\vec x_m^{(k)}} = \lambda_1 + O \left( \left| \dfrac{\lambda_2}{\lambda_1} \right|^k \right). \]

	Значить $\mu_1^{(k)} \xrightarrow[k\to\infty]{} \lambda_1$. \\

	Якщо матриця $A = A^T$ симетрична, то існує ортонормована система векторів $(\vec e_i, \vec e_j) = \delta_{i,j}$. Тому

	\begin{multline*}
		\mu_1^{(k)} = \dfrac{\left(\vec x^{(k+1)}, \vec x^{(k)}\right)}{\left(\vec x^{(k)}, \vec x^{(k)}\right)} = \dfrac{\left(\Sum_{i=1}^n c_i \lambda_i^{k+1} \vec e_i,\Sum_{j=1}^n c_j \lambda_j^k \vec e_j\right)}{\left(\Sum_{i=1}^n c_i \lambda_i^k \vec e_i,\Sum_{j=1}^n c_j \lambda_j^k \vec e_j\right)} = \dfrac{\Sum_i c_i^2 \lambda_i^{2k+1}}{\Sum_i c_i^2 \lambda_i^{2k}} = \\
		= \dfrac{c_1^2\lambda_1^{2k+1}+c_2^2\lambda_2^{2k+1}+\ldots}{c_1^2\lambda_1^{2k}+c_2^2\lambda_2^{2k}+\ldots} = \lambda_1 + O \left( \left| \dfrac{\lambda_2}{\lambda_1} \right|^{2k} \right) \xrightarrow[k\to\infty]{} \lambda_1.
	\end{multline*}

	Це означає збіжність до максимального за модулем власного значення з квадратичною швидкістю.\\

	Якщо $\lambda_1 > 1$, то при проведенні ітерацій відбувається зріст компонент вектора $\vec x^{(k)}$, що приводить до ``переповнення'' (overflow). Якщо ж $\lambda_1 < 1$, то це приводить до зменшення компонент (underflow). Позбутися негативу такого явища можна нормуючи вектори $\vec x^{(k)}$ на кожній ітерації. \\

	\textbf{Алгоритм} степеневого методу знаходження максимального за модулем власного значення з точністю $\epsilon$ виглядає так:
	\begin{enumerate}
		\item $\vec x^{(0)} \to \vec e_0 = \frac{\vec x^{(0)}}{\left\|\vec x^{(0)}\right\|}$;

		\item $\vec x^{(k+1)} = A \vec x^{(k)}$, $\mu_1^{(k)} = \left(\vec x^{(k+1)}, \vec e^{(k)}\right)$, $\vec e^{(k+1)} = \frac{\vec x^{(k+1)}}{\left\|\vec x^{(k+1)}\right\|}$, $k = 0,1,\ldots$;

		\item $|\mu_1^{(k+1)} - \mu_1^{(k)}| \ge \epsilon$ \verb|goto| (б);

		\item $\lambda_1 \approx \mu_1^{(k+1)}$.
	\end{enumerate}

	За цим алгоритмом для симетричної матриці $A^T = A$ швидкість прямування $\mu_1^{(k)}$ до $\lambda_{\max}$ -- квадратична.

	\item \textit{Знаходження} $\lambda_2: |\lambda_1| > |\lambda_2| > |\lambda_3| \ge \ldots$. Нехай $\lambda_1$, $\vec e_1$ відомі.

	\begin{problem}
		Довести, що якщо $|\lambda_1| > |\lambda_2| > |\lambda_3| \ge \ldots$, то
		\[ \mu_2^{(k)} = \dfrac{\vec x_m^{(k+1)} - \lambda_1 \vec x_m^{(k)}}{\vec x_m^{(k)} - \lambda_1 \vec x_m^{(k-1)}} \xrightarrow[k\to\infty]{} \lambda_2,\]
		де $\vec x^{(k+1)} = A \vec x^{(k)}$, $x_m^{(k)}$ -- $m$-та компонента $\vec x^{(k)}$.
	\end{problem}

	\begin{problem}
		Побудувати алгоритм обчислення $\lambda_2$, $\vec e_2$, використовуючи нормування векторів та скалярні добутки для обчислення $\mu_2^{(k)})$.
	\end{problem}	

	\item \textit{Знаходження мінімального власного числа} $\lambda_{\min}(A) = \min_i |\lambda_i(A)|$.

	Припустимо , що $\forall i$: $\lambda_i(A) > 0$ та відоме $\lambda_{\max}$. Розглянемо матрицю $B = \lambda_{\max}\cdot E - A$. Маємо \[ \forall i: \lambda_i(B) = \lambda_{\max} - \lambda_i(A). \]

	Тому $\Max_i \lambda_i (B) = \lambda_{\max} - \Min_i \lambda_i(A)$. Звідси $\lambda_{\min}(A) = \lambda_{\max}(A) - \lambda_{\max}(B)$. \\

	Якщо $\exists i: \lambda_i(A) < 0$, то будуємо матрицю $ \overline{A} = \sigma \cdot E + A$, $\sigma > 0$, $\overline{A} > 0$ і для неї попередній розгляд дає необхідний результат. Замість $\lambda_{\max}$ в матриці $B$ можна використовувати $\|A\|$. \\

	Ще один спосіб обчислення мінімального власного значення полягає в використання обернених ітерацій:
	\begin{equation}
		\label{eq:5.3}
		A \vec x^{(k+1)} = \vec x^k, \quad k = 0, 1, \ldots
	\end{equation}

	Але цей метод вимагає більшої кількості арифметичних операцій: складність методу на основі формули (\ref{eq:5.2}) $Q = O(n^2)$ , а на основі (\ref{eq:5.3}) -- $Q = O(n^3)$, оскільки треба розв'язувати СЛАР, але збігається метод (\ref{eq:5.3}) швидше.
\end{enumerate}

\subsection{Ітераційний метод обертання}
Це метод розв'язання повної проблеми власних значень для симетричних матриць $A^T = A$. Існує матриця $U$, що приводить матрицю $A$ до діагонального виду:
\begin{equation}
	\label{eq:5.4}
	A = U \Lambda U^T,
\end{equation}
де $\Lambda$ -- діагональна матриця, по діагоналі якої стоять власні значення $\lambda_i$; $U$ -- унітарна матриця, тобто: $U^{-1}=U^T$. \\

З (\ref{eq:5.4}) маємо
\begin{equation}
	\label{eq:5.5}
	\Lambda = U^T \Lambda U,
\end{equation}
Нехай $\exists \tilde U$ -- матриця, така що $\tilde \Lambda = \tilde U^T A \tilde U$ і $\tilde \Lambda = \left(\tilde \lambda_{i,j}\right)_{i,j=1}^n$, $\left|\tilde \lambda_{i,j}\right| < \delta \ll 1$, $i \ne j$. \\

Тоді діагональні елементи мало відрізняються від власних значень \[ |\tilde \lambda_{i,i} - \lambda_i(A) | < \epsilon = \epsilon(\delta). \]

Введемо 
\[t(A) = \Sum_{\substack{i, j = 1 \\ i \ne j}}^n a_{i,j}^2.\]
З малості величини $t(A)$ витікає, що діагональні елементи малі. По $A = A_0$ за допомогою матриць обертання 
\[ U_k = \begin{pmatrix} 1 & \cdots & 0 & \cdots & 0 & \cdots & 0 \\ \vdots & \ddots & \vdots & \ddots & \vdots & \ddots & \vdots \\ 0 & \cdots & \cos \phi & \cdots & -\sin \phi & \cdots & 0 \\ \vdots & \ddots & \vdots & \ddots & \vdots & \ddots & \vdots \\ 0 & \cdots & \sin \phi & \cdots & \cos \phi & \cdots & 0 \\ \vdots & \ddots & \vdots & \ddots & \vdots & \ddots & \vdots \\ 0 & \cdots & 0 & \cdots & 0 & \cdots & 1 \\ \end{pmatrix}. \]
що повертають систему векторів на кут $\phi$, побудуємо послідовність $\{A_k\}$ таку, що $A_k \to \Lambda$ при $k\to\infty$.

\begin{problem}
	Показати, що матриця обертання $U_k$ є унітарною: $U_k^{-1} = U_k^T$.
\end{problem}

Послідовно будуємо:
\begin{equation}
	\label{eq:5.6}
	A_{k+1} = U_K^T A_k U_k,
\end{equation}
Процес (\ref{eq:5.6}) називається \textit{монотонним}, якщо: $t(A_{k+1})<t(A_k)$.

\begin{problem}
Довести, що для матриці (\ref{eq:5.6}) виконується:
\begin{equation}
	\label{eq:5.7}
	a_{i,j}^{(k+1)} = a_{i,j}^{(k)} \cos (2 \phi) + \dfrac12 \left(a_{j,j}^{(k)} - a_{i,i}^{(k)}\right) \sin (2 \phi),
\end{equation}

Показати, що $t(A_{k+1})=t(A_k)-2\left(a_{i,j}^{(k)}\right)^2$, якщо вибирати $\phi$ з умови $a_{i,j}^{(k+1)} = 0$.
\end{problem}

Звідси 
\[\phi = \phi_k = \frac12 \arctan\left(p^{(k)}\right), \quad p^{(k)} = \frac{2a_{i,j}^{(k)}}{a_{i,i}^{(k)}-a_{j,j}^{(k)}},\] де $\left|a_{i,j}^{(k)}\right| = \Max_{m \ne l} \left|a_{m,l}^{(k)}\right|$. Тоді
$t(A_k) \xrightarrow[k\to\infty]{} 0$. Чим більше $n$ тим більше ітерацій необхідно для зведення $A$ до $\Lambda$. \\

Якщо матриця несиметрична, то застосовують $QR$, $QL$ методи.