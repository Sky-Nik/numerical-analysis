\section{Аналіз похибок заокруглення}

\subsection{Види похибок}

Нехай необхідно розв’язати рівняння
\begin{equation}
	\label{eq:1.1}
	Au = f.
\end{equation}
За рахунок неточно заданих вхідних даних насправді ми маємо рівняння
\begin{equation}
	\label{eq:1.2}
	\tilde A \tilde u = \tilde f.
\end{equation}
Назвемо $\delta_1 = u - \tilde u$ -- \textit{неусувною похибкою}. \\

Застосування методу розв‘язання \eqref{eq:1.2} приводить до рівняння
\begin{equation}
	\label{eq:1.3}
	\tilde A_h \tilde u_h = \tilde f_h,
\end{equation}
де $h > 0$ -- малий параметр. Назвемо $\delta_2 \tilde u - \tilde u_h$ -- \textit{похибкою методу}. \\

Реалізація методу на ЕОМ приводить до рівняння
\begin{equation}
	\label{eq:1.4}
	\tilde A_h^* \tilde u_h^* = \tilde f_h^*.
\end{equation}
Назвемо $\delta_3 = \tilde u_h^* - \tilde u_h$ -- \textit{похибкою заокруглення}. \\

Тоді \textit{повна похибка} $\delta = u - \tilde u_h^* = \delta_1 + \delta_2 + \delta_3$. \\

\begin{definition}
	Кажуть, що задача \eqref{eq:1.1} \textit{коректна}, якщо
	\begin{enumerate}
		\item $\forall f \in F$ $\exists! u \in U$;
		\item задача \eqref{eq:1.1} \textit{стійка}, тобто
		\begin{equation}
			\label{eq:1.5}
			\forall \epsilon > 0 \quad \exists \delta > 0: \|A-\tilde A\| < \delta, \|f-\tilde f\| < \delta \Rightarrow \|u - \tilde u\| < \epsilon.
		\end{equation}
	\end{enumerate}
\end{definition}

Якщо задача \eqref{eq:1.1} \textit{некоректна}, то або розв‘язок її не існує, або він неєдиний, або він нестійкий, тобто 
\begin{equation}
	\label{eq:1.6}
	\exists \epsilon > 0: \forall \delta > 0: \exists A, f: \| A - \tilde A\|<\delta, \|f-\tilde f\| < \delta, \|u-\tilde u\| > \epsilon.
\end{equation}

\textit{Абсолютна похибка} $\Delta (x^*) \ge \max_x |x - x^*|$. \\

\textit{Відносна похибка} $\delta (x^*) \ge \max_x \Delta (x^*) / |x^*|$. \\

\textit{Значущими цифрами} називаються всі цифри, починаючи з першої ненульової зліва. \\

\textit{Вірна цифра} -- це значуща, якщо абсолютна похибка за рахунок відкидання всіх молодших розрядів не перевищує одиниці розряду цієї цифри. Тобто, якщо 
\begin{equation}
	\label{eq:1.7}
	x^* = \overline{\alpha_n \ldots \alpha_0.\alpha_{-1}\ldots\alpha_{-p}\ldots},
\end{equation}
то $\alpha_{-p}$ -- вірна, якщо $\Delta (x^*) \le 10^{-p}$. \\

Інколи $\Delta (x^*) \le w \cdot 10^{-p}$, $1/2 \le w < 1$, наприклад, $w = 0.55$.

\subsection{Підрахунок похибок в ЕОМ}

Підрахуємо відносну похибку заокруглення числа $x$ на ЕОМ з плаваючою комою. В $\beta$-ічній системі числення число представляється у вигляді
\begin{equation}
	\label{eq:1.8}
	x = \pm (\alpha_1 \beta^{-1} + \alpha_2 \beta^{-2} + \ldots + \alpha_t \beta^{-t} + \ldots) \beta^p,
\end{equation}
де $0 \le \alpha_k < \beta$, $\alpha_1 \ne 0$, $k = 1,2,\ldots$ \\

Якщо в ЕОМ $t$ розрядів, то при відкиданні молодших розрядів ми оперуємо з наближеним значенням 
\begin{equation}
	\label{eq:1.9}
	x^* = \pm (\alpha_1 \beta^{-1} + \alpha_2 \beta^{-2} + \ldots + \alpha_t \beta^{-t}) \beta^p,
\end{equation}
і відповідно похибка заокруглення 
\begin{equation}
	\label{eq:1.9_1}
	x - x^* = \pm \beta^p (\alpha_{t+1} \beta^{-t-1} + \ldots)
\end{equation}
Тоді її можна оцінити так
\begin{multline}
	\label{eq:1.10}
	|x - x^*| \le \beta^{p-t-1} \cdot (\beta-1) \cdot (1 + \beta^{-1}+\ldots)\le\\
	\le \beta^{p-t-1} \cdot (\beta-1) \cdot \dfrac{1}{1-\beta^{-1}}=\beta^{p-t}.
\end{multline}

Якщо в представлені \eqref{eq:1.8} взяти $\alpha_1 = 1$, то $|x| \ge \beta^p \cdot \beta^{-1}$. Звідси остаточно 
\begin{equation}
	\label{eq:1.11}
	\delta (x^*) \le \dfrac{\beta^{p-t}}{\beta^{p-1}}=\beta^{1-t}.
\end{equation}

При більш точних способах заокруглення можна отримати оцінку $\delta (x^*) \le \beta^{1-t} / 2 = \epsilon$. Число $\epsilon$ називається ``машинним іпсилон''. Наприклад, для $\beta = 2$, $t = 24$, $\epsilon = 2^{-24} \approx 10^{-7}$.

\subsection{Підрахунок похибок обчислення значення функції}

Нехай задана функція 
\begin{equation}
	\label{eq:1.11_1}
	y = f(x_1, \ldots, x_n) \in C^1(\Omega)
\end{equation}
Необхідно обчислити її значення при наближеному значенні аргументів 
\begin{equation}
	\label{eq:1.11_2}
	\vec x^* = (x_1^*, \ldots, x_n^*),
\end{equation}
де $|x_i - x_i^*| \le \Delta (x_i^*)$ та оцінити похибку обчислення значення функції
\begin{equation}
	\label{eq:1.11_3}
	y^* = f(x_1^*, \ldots, x_n^*).
\end{equation}

Маємо 
\begin{equation}
	\label{eq:1.12}
	|y-y^*| = |f\left(\vec x\right) - f\left(\vec x^*\right)| = \left| \Sum_{i=1}^n \dfrac{\partial f}{\partial x_i} \left(\vec \xi\right) (x_i - x_i^*) \right| \le \Sum_{i=1}^n B_i \cdot \Delta (x_i^*), 
\end{equation}
де 
\begin{equation}
	\label{eq:1.13}
	B_i = \Max_{\vec x \in U} \left| \dfrac{\partial f}{\partial x_i}\left(\vec x\right) \right|.
\end{equation}

Тут 
\begin{equation}
	\label{eq:1.14}
	U = \left\{ \vec x = (x_1, \ldots, x_n): |x_i - x_i^*| \le \Delta (x_i^*)\right\} \in \Omega, \quad i=\overline{1,n}.
\end{equation}
Отже
\begin{equation}
	\label{eq:1.15}
	\Delta (y^*) = |y - y^*| \prec \Sum_{i=1}^n n_i \cdot \Delta (x_i^*),
\end{equation}
з точністю до величин першого порядку малості по
\begin{equation}
	\label{eq:1.15_1}
	\Delta (x^*) = \Max_{i=\overline{1,n}} \Delta (x_i^*),
\end{equation}
де
\begin{equation}
	\label{eq:1.16}
	b_i = \left| \dfrac{\partial f}{\partial x_i}\left(\vec x^*\right) \right|
\end{equation}
та ``$\prec$'' означає приблизно менше. \\

\subsubsection{Похибки арифметичних операцій}

\begin{enumerate}
	\item Сума: $y = x_1 + x_2$, $x_1, x_2 > 0$, 
	\begin{equation}
		\label{eq:1.17}
		\begin{aligned}
		\Delta (y^*) &\le \Delta (x_1^*) + \Delta (x_2^*), \\ 
		\delta (y^*) &\le \dfrac{\Delta (x_1^*) + \Delta (x_2^*)}{x_1^* + x_2^*} \le \max\{\delta (x_1^*), \delta (x_2^*)\}.
		\end{aligned}
	\end{equation}
	
	\item Різниця: $y = x_1 - x_2$, $x_1 > x_2 > 0$,
	\begin{equation}
		\label{eq:1.18}
		\begin{aligned}
		\Delta (y^*) &\le \Delta (x_1^*) + \Delta (x_2^*), \\
		\delta (y^*) &\le \dfrac{x_2^* \delta (x_1^*) + (x_1^*) \delta (x_2^*)}{x_1^* - x_2^*}.
		\end{aligned}
	\end{equation}
	
	При близьких $x_1^*$, $x_2^*$ зростає відносна похибка (за рахунок втрати вірних цифр).

	\item Добуток: $y = x_1 \cdot x_2$, $x_1, x_2 > 0$,
	\begin{equation}
		\label{eq:1.19}
		\begin{aligned}
		\Delta (y^*) &\prec x_2^* \Delta (x_1^*) + x_1^* \Delta (x_2^*), \\
		\delta (y^*) &\le \delta (x_1^*) + \delta (x_2^*).
		\end{aligned}
	\end{equation}

	\item Частка: $y = \dfrac{x_1}{x_2}$, $x_1, x_2 > 0$,
	\begin{equation}
		\label{eq:1.20}
		\begin{aligned}
		\Delta (y^*) &\prec \dfrac{x_2^* \Delta (x_1^*) + x_1^* \Delta (x_2^*)}{(x_2^*)^2}, \\
		\delta (y^*) &\le \delta (x_1^*) + \delta (x_2^*).
		\end{aligned}
	\end{equation}

	При малих $x_2^*$ зростає абсолютна похибка (за рахунок зростання результату ділення). 
\end{enumerate}

\subsection{Обернена задача аналізу позибок}

Нагадаємо, що \textit{пряма задача} аналізу похибок полягає у обчисленні $\Delta (y^*), \delta (y^*)$ по заданих $\Delta (x_i^*)$, $i = \overline{1, n}$. \\

\textit{Обернена задача} полягає у знаходженні $\Delta (x_i^*)$, $i = \overline{1, n}$ по заданих $\Delta (y^*)$, $\delta (y^*)$. Якщо $n > 1$, маємо одну умову 
\begin{equation}
	\label{eq:1.21}
	\Sum_{i=1}^n b_i \Delta (x_i^*) < \epsilon
\end{equation}
для багатьох невідомих $\Delta (x_i^*)$. Вибирають їх із однієї з умов 
\begin{equation}
	\label{eq:1.22}
	b_i \Delta (x_i^*) < \dfrac{\epsilon}{n},
\end{equation}
або
\begin{equation}
	\label{eq:1.23}
	\Delta (x_i^*) < \dfrac{\epsilon}{\Sum_{i=1}^n b_i}.
\end{equation}