\section{Аналіз похибок заокруглення}

\subsection{Види похибок}

Нехай необхідно розв’язати рівняння
\begin{equation}
	\label{eq:1.1}
	Au = f.
\end{equation}
За рахунок неточно заданих вхідних даних насправді ми маємо рівняння
\begin{equation}
	\label{eq:1.2}
	\tilde A \tilde u = \tilde f.
\end{equation}
Назвемо $\delta_1 = u - \tilde u$ -- \textit{неусувною похибкою}. \\

Застосування методу розв‘язання (\ref{eq:1.2}) приводить до рівняння
\begin{equation}
	\label{eq:1.3}
	\tilde A_h \tilde u_h = \tilde f_h,
\end{equation}
де $h > 0$ -- малий параметр. Назвемо $\delta_2 \tilde u - \tilde u_h$ -- \textit{похибкою методу}. \\

Реалізація методу на ЕОМ приводить до рівняння
\begin{equation}
	\label{eq:1.4}
	\tilde A_h^* \tilde u_h^* = \tilde f_h^*.
\end{equation}
Назвемо $\delta_3 = \tilde u_h^* - \tilde u_h$ -- \textit{похибкою заокруглення}. \\

Тоді \textit{повна похибка} $\delta = u - \tilde u_h^* = \delta_1 + \delta_2 + \delta_3$. \\

\begin{definition}
	Кажуть, що задача (\ref{eq:1.1}) \textit{коректна}, якщо
	\begin{enumerate}
		\item $\forall f \in F$ $\exists! u \in U$;
		\item задача (\ref{eq:1.1}) \textit{стійка}, тобто
		\[ \forall \epsilon > 0 \quad \exists \delta > 0: \|A-\tilde A\| < \delta, \|f-\tilde f\| < \delta \Rightarrow \|u - \tilde u\| < \epsilon. \]
	\end{enumerate}
\end{definition}

Якщо задача (\ref{eq:1.1}) \textit{некоректна}, то або розв‘язок її не існує, або він неєдиний, або він нестійкий, тобто \[ \exists \epsilon > 0: \forall \delta > 0: \| A - \tilde A\|<\delta, \|f-\tilde f\| < \delta \Rightarrow \|u-\tilde u\| > \epsilon.\]

\textit{Абсолютна похибка} $\Delta x \le |x - x^*|$. \\

\textit{Відносна похибка} $\delta x \le \dfrac{\Delta x}{|x|}$, або $\dfrac{\Delta x}{|x^*|}$. \\

\textit{Значущими цифрами} називаються всі цифри, починаючи з першої ненульової зліва. \\

\textit{Вірна цифра} -- це значуща, якщо абсолютна похибка за рахунок відкидання всіх молодших розрядів не перевищує одиниці розряду цієї цифри. Тобто, якщо \[x^* = \overline{\alpha_n \ldots \alpha_0.\alpha_{-1}\ldots\alpha_{-p}\ldots},\] то $\alpha_{-p}$ -- вірна, якщо $\Delta x \le 10^{-p}$ (інколи $\Delta x \le w \cdot 10^{-p}$, $\frac12 \le w < 1$, наприклад, $w = 0.55$).

\subsection{Підрахунок похибок в ЕОМ}

Підрахуємо відносну похибку заокруглення числа $x$ на ЕОМ з плаваючою комою. В $\beta$-ічній системі числення число представляється у вигляді
\begin{equation}
	\label{eq:1.5}
	x = \pm (\alpha_1 \beta^{-1} + \alpha_2 \beta^{-2} + \ldots + \alpha_t \beta^{-t} + \ldots) \beta^p,
\end{equation}
де $0 \le \alpha_k < \beta$, $\alpha_1 \ne 0$, $k = 1,2,\ldots$. \\

Якщо в ЕОМ $t$ розрядів, то при відкиданні молодших розрядів ми оперуємо з наближеним значенням \[x^* = \pm (\alpha_1 \beta^{-1} + \alpha_2 \beta^{-2} + \ldots + \alpha_t \beta^{-t}) \beta^p,\] 
і відповідно похибка заокруглення $x - x^* = \pm \beta^p (\alpha_{t+1} \beta^{-t-1} + \ldots)$. Тоді її можна оцінити так \[ |x - x^*| \le \beta^{p-t-1}(\beta-1)(1 + \beta^{-1}+\ldots)\le \beta^{p-t-1}(\beta-1)\dfrac{1}{1-\beta^{-1}}=\beta^{p-t}.\]

Якщо в представлені (\ref{eq:1.5}) взяти $\alpha_1 = 1$, то $|x| \ge \beta^p \beta^{-1}$. Звідси остаточно \[\delta x \le \dfrac{\beta^{p-t}}{\beta^{p-1}}=\beta^{1-t}.\]

При більш точних способах заокруглення можна отримати оцінку $\delta x \le \frac12 \beta^{1-t} = \epsilon$. Число $\epsilon$ називається ``машинним іпсилон''. Наприклад, для $\beta = 2$, $t = 24$, $\epsilon = 2^{-24} \approx 10^{-7}$.

\subsection{Підрахунок похибок обчислення значення функції}

Нехай задана функція $y = f(x_1, \ldots, x_n) \in C^1(\Omega)$. Необхідно обчислити її значення при наближеному значенні аргументів $\vec x^* = (x_1^*, \ldots, x_n^*)$, де $|x_i - x_i^*| \le \Delta x_i$ та оцінити похибку обчислення значення функції $y^* = f(x_1^*, \ldots, x_n^*)$. Маємо 
\[ |y-y^*| = |f\left(\vec x\right) - f\left(\vec x^*\right)| = \left| \Sum_{i=1}^n \dfrac{\partial f}{\partial x_i} \left(\vec \xi\right) (x_i - x_i^*) \right| \le \Sum_{i=1}^n B_i \cdot \Delta x_i, \] 
де $B_i = \Max_{\vec x \in U} \left| \dfrac{\partial f}{\partial x_i}\left(\vec x\right) \right|$. \\

Тут $U = \left\{ \vec x = (x_1, \ldots, x_n): |x_i - x_i^*| \le \Delta x_i\right\} \in \Omega$, $i=\overline{1,n}$. Отже з точністю до величин першого порядку малості по $\Delta x = \Max_{i=\overline{1,n}} \Delta x_i$, 
\[ \Delta y = |y - y^*| \prec \Sum_{i=1}^n n_i \cdot \Delta x_i,\]
де $b_i = \left| \dfrac{\partial f}{\partial x_i}\left(\vec x^*\right) \right|$ та ``$\prec$'' означає приблизно менше. \\

Розглянемо похибки арифметичних операцій.
\begin{enumerate}
	\item Сума: $y = x_1 + x_2$, $x_1, x_2 > 0$, 
	\[ \Delta y \le \Delta x_1 + \Delta x_2, \quad \delta y \le \dfrac{\Delta x_1 + \Delta x_2}{x_1 + x_2} \le \max(\delta x_1, \delta x_2). \] 
	
	\item Різниця: $y = x_1 - x_2$, $x_1 > x_2 > 0$,
	\[ \Delta y \le \Delta x_1 + \Delta x_2, \quad \delta y \le \dfrac{x \delta x_1 + x_1 \delta x_2}{x_1 - x_2}. \]
	
	При близьких $x_1$, $x_2$ зростає відносна похибка (за рахунок втрати вірних цифр).

	\item Добуток: $y = x_1 \cdot x_2$, $x_1, x_2 > 0$,
	\[ \Delta y \prec x_2 \Delta x_1 + x_1 \Delta x_2, \quad \delta y \le \delta x_1 + \delta x_2. \]

	\item Частка: $y = \dfrac{x_1}{x_2}$, $x_1, x_2 > 0$,
	\[ \Delta y \prec \dfrac{x_2 \Delta x_1 + x_1 \Delta x_2}{x_2^2}, \quad \delta y \le \delta x_1 + \delta x_2. \]

	При малих $x_2$ зростає абсолютна похибка (за рахунок зростання результату ділення). 
\end{enumerate}

\textit{Пряма задача} аналізу похибок: обчислення $\Delta y, \delta y$ по заданих $\Delta x_i$, $i = \overline{1, n}$. \\

\textit{Обернена задача}: знаходження $\Delta x_i$, $i = \overline{1, n}$ по заданих $\Delta y$, $\delta y$. Якщо $n > 1$, маємо одну умову 
\[\Sum_{i=1}^n b_i \Delta x_i < \epsilon\]
для багатьох невідомих $\Delta x_i$. Вибирають їх із однієї з умов 
\[ \forall i: b_i \Delta x_i < \dfrac{\epsilon}{n} \quad \text{або} \quad \Delta x_i < \dfrac{\epsilon}{\Sum_{i=1}^n b_i}. \]