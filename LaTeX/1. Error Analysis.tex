\documentclass[a4paper, 12pt]{article}
\usepackage[utf8]{inputenc}
\usepackage[english, ukrainian]{babel}
\usepackage{amsmath, amssymb}
\usepackage[top = 1 cm, left = 1 cm, right = 1 cm, bottom = 1 cm]{geometry} 

\usepackage{multicol, graphicx}

\usepackage{amsthm}
\theoremstyle{definition}
\newtheorem{problem}{Задача}
\newtheorem{definition}{Визначення}

\allowdisplaybreaks
\setlength\parindent{0pt}
\pagestyle{empty}

\newcommand{\argmax}{\arg\max}
\newcommand{\argmin}{\arg\min}

\newcommand{\dif}{\mathrm{d}}
\newcommand{\dydx}{\dfrac{\dif y}{\dif x}}
\newcommand{\dxdt}{\dfrac{\dif x}{\dif t}}
\newcommand{\dydt}{\dfrac{\dif y}{\dif t}}

\newcommand{\NN}{\mathbb{N}} 
\newcommand{\ZZ}{\mathbb{Z}}
\newcommand{\QQ}{\mathbb{Q}}
\newcommand{\RR}{\mathbb{R}}
\newcommand{\CC}{\mathbb{C}}

\newcommand{\LaReF}[1]{(\ref{#1})}

\renewcommand{\epsilon}{\varepsilon}
\renewcommand{\phi}{\varphi}

\newcommand{\ws}{\text{ }}

\newcommand{\Max}{\displaystyle\max\limits}
\newcommand{\Min}{\displaystyle\min\limits}
\newcommand{\Sum}{\displaystyle\sum\limits}
\newcommand{\Int}{\displaystyle\int\limits}
\newcommand{\Prod}{\displaystyle\prod\limits}


\begin{document}

\section{Аналіз похибок заокруглення}
	
\subsection{Види похибок}

Нехай необхідно розв'язати рівняння \begin{equation} \label{eq:au=f} Au = f. \end{equation}

Неточно задані вхідні дані призводять до рівняння \begin{equation} \label{eq:au=f 2} \tilde A \tilde u = \tilde f.\end{equation}

Назвемо $\delta_1 = u - \tilde u$ \textit{незсувною похибкою}.\\

Застосування методу розв'язання \LaReF{eq:au=f 2} призводить до рівняння \begin{equation} \label{eq:au=f 3} \tilde A_h \tilde u_h = \tilde f_h, \end{equation} де $h > 0$ -- малий параметр. Назвемо $\delta_2 = \tilde u - \tilde u_h$ \textit{похибкою методу}.\\

Реалізація методу на ЕОМ призводить до рівняння \begin{equation} \label{eq:au=f 4} \tilde A_h^\star \tilde u_h^\star = \tilde f_h^\star. \end{equation} 

Назвемо $\delta_3 = \tilde u_h - \tilde u_h^\star$ \textit{похибкою заокруглення}. \\

Назвемо $\delta = u - \tilde u_h^\star = \delta_1 + \delta_2 + \delta_3$ \textit{повною похибкою}.

\begin{definition}
	Кажуть, що задача \LaReF{eq:au=f} \textit{коректна}, якщо 
	\begin{enumerate}
		\item $\forall f\in F$ $\exists! u\in U$;
		\item задача \LaReF{eq:au=f} \textit{стійка}, тобто \[\forall \epsilon > 0 \ws \exists \delta > 0 \ws \forall f: \left\|A-\tilde A\right\|<\delta, \left\|f-\tilde f\right\|<\delta \Rightarrow \left\|u-\tilde u\right\|<\epsilon.\]
	\end{enumerate}
\end{definition}

Якщо задача \LaReF{eq:au=f} некоректна, то або розв'язок її не існує, або він не єдиний, або він нестійкий, тобто \[\exists \epsilon > 0 \ws \forall \delta > 0 \ws \exists f: \left\|A-\tilde A\right\|<\delta, \left\|f-\tilde f\right\|<\delta, \left\|u-\tilde u\right\|>\epsilon.\]

\textit{Абсолютна похибка} $\Delta x\le |x-x^\star|$.\\

\textit{Відносна похибка} $\delta x\le \dfrac{\Delta x}{|x|}$ або $\dfrac{\Delta x}{|x^\star|}$.\\

\textit{Значущими цифрами} називаються всі цифри,починаючи з першої ненульової зліва.\\ 

\textit{Вірна цифра} -- це значуща, якщо абсолютна похибка за рахунок відкидання всіх молодших розрядів не перевищує одиниці розряду цієї цифри. Тобто, якщо $x^\star = \overline{\alpha_n\ldots\alpha_0}.\overline{\alpha_{-1}\ldots\alpha_{-p}}$, то $a_{-p}$ -- вірна, якщо $\Delta x\le 10^{-p}$ (інколи беруть $\Delta x\le q\cdots 10^{-p}$, $1/2\le w<1$, наприклад $w=0.55$).

\subsection{Підрахунок похибок в ЕОМ}

Обчислимо відносну похибку заокруглення числа $x$ на ЕОМ з плавоючою комою. У системі числення з основою $\beta$ число $x$ представляється у вигляді \begin{equation} \label{eq:x-beta-basis} x=\pm\left(\alpha_1\beta^{-1}+\alpha_2\beta^{-2}+\ldots+\alpha_t\beta^{-t}+\ldots\right)\beta^p, \end{equation} де $0\le\alpha_k<\beta$, $\alpha_1\ne0$, $k=1,2,\ldots$\\

Якщо в ЕОМ $t$ розрядів, то при відкиданні молодших розрядів ми оперуємо з наближеним значенням \[x^\star = \pm\left(\alpha_1\beta^{-1}+\alpha_2\beta^{-2}+\ldots+\alpha_t\beta^{-t}\right)\beta^p\] і, відповідно, похибка заокруглення $x-x^\star = \pm \beta^p\left(\alpha_{t+1}\beta^{-t-1}+\ldots\right)$. Її можна оцінити так: \[|x-x^\star|\le \beta^{p-t-1}(\beta-1)(1+\beta^{-1}+\ldots)\le \beta^{p-t-1}(\beta - 1)\dfrac1{1-\beta^{-1}}=\beta^{p-t}.\]

Враховуючи, що $\alpha_1\ne 0$, маємо $|x|\ge \beta^p\beta^{-1}=\beta^{p-1}$. Звідси остаточно \[\delta x\le \dfrac{\beta^{p-t}}{\beta^{p-1}}=\beta^{-t+1}.\]

При точніших способах заокруглення можна отримати оцінку $\delta x\le \beta^{-t+1}/2=\epsilon$. Число $\epsilon$ називаєтсья \textit{машинним іпсилон}. Наприклад, для $\beta=2$, $t=24$, $\epsilon=2^{-24}\approx 10^{-7}$. 

\subsection{Обчислення похибок обчислення значення функції}

Нехай задана функція $y=f(x_1,\ldots,x_n)\in C^{(1)}(\Omega)$. Необхідно обчислити її значення при наближеному значенні агрументів $\vec x^\star = \left(x_1^\star,\ldots,x_n^\star\right)$, де $\left|x_i-x_i^\star\right|\le\Delta x_i$ та оцінити похибку обчислення значення функції $y^\star = f\left(x_1^\star,\ldots,x_n^\star\right)$.\\

 Маємо \[\left|y-y^\star\right|=\left|f(\vec x)-f(\vec x^\star)\right|=\left|\Sum_{i=1}^n \dfrac{\partial f}{\partial x_i}(\vec \xi)(x_i-x_i^\star)\right|\le \Sum_{i=1}^n B_i\cdots \Delta x_i,\] де $B_i=\Max_{\vec x\in U} \left|\dfrac{\partial f}{\partial x_i}(\vec x)\right|$, $U=\left\{\vec x=(x_1,\ldots,x_n):\left|x_i-x_i^\star\right|\le\Delta x_i, i=1,\ldots,n\right\}\subset\Omega$.\\

 Отже, з точністю до величин першого порядку малості по $\Delta x=\Max_i \Delta x_i$, $\Delta y=|y-y^\star|\prec \Sum_{i=1}^n b_i\cdots \Delta x_i$, де $b_i=\left|\dfrac{\partial f}{\partial x_i}\left(\vec x^\star\right)\right|$ і ``$\prec$'' означає \textit{приблизно менше}.\\

Розглянемо похибки арифметичних операцій.
\begin{enumerate}
	\item Сума: $y=x_1+x_2$, $x_1,x_2>0$, $\Delta y\le \Delta x_1+\Delta x_2$, $\delta y=\dfrac{\Delta y}{y}=\dfrac{\Delta y}{x_1+x_2}\le \dfrac{\Delta x_1 + \Delta x_2}{x_1+x_2}\le\max(\delta x_1,\delta x_2)$.
	\item Різниця: $y=x_1-x_2$, $x_1>x_2>0$, $\Delta y\le \Delta x_1+\Delta x_2$, $\delta y=\dfrac{\Delta y}{y}=\dfrac{\Delta y}{x_1-x_2}\le\dfrac{\Delta x_1+\Delta x_2}{x_1-x_2}=\dfrac{x_1\delta x_1+x_2\delta x_2}{x_1-x_2}$. \\ Як бачимо, при близьких аргументах зростає відносна похибка.
	\item Добуток: $y=x_1\cdot x_2$, $x_1,x_2>0$, $\Delta y=x_1\Delta x_2+x_2\Delta x_1+\Delta x_1\Delta x_2\prec x_1\Delta x_2+x_2\Delta x_1$, $\delta y=\dfrac{\Delta y}{y}=\dfrac{\Delta y}{x_1x_2}\prec\dfrac{x_1\Delta x_2+x_2\Delta x_1}{x_1x_2}=\dfrac{\Delta x_1}{x_1}+\dfrac{\Delta x_2}{x_2}=\delta x_1+\delta x_2$.
	\item Частка: $y=\dfrac{x_1}{x_2}$, $x_1,x_2>0$, $\Delta y=\dfrac{x_2\Delta x_1-x_1\Delta x_2}{x_2(x_2+\Delta x_2)}<\dfrac{x_2\Delta x_1+x_1\Delta x_2}{x_2(x_2+\Delta x_2)}\prec \dfrac{x_2\Delta x_1+x_1\Delta x_2}{x_2^2}$, $\delta y=\dfrac{\Delta y}{y}=\dfrac{x_2\Delta y}{x_1}\prec \dfrac{x_2\Delta x_1+x_1\Delta x_2}{x_1x_2}=\dfrac{\Delta x_1}{x_1}+\dfrac{\Delta x_2}{x_2}=\delta x_1+\delta x_2$. \\ Як бачимо, при малих $x_2$ зростає абсолютна похибка.
\end{enumerate}

\textit{Пряма задача} аналізу похибок: обчислення $\Delta y$, $\delta y$ за заданими $\Delta x_i,i=1,\ldots,n$.\\

\textit{Обернена задача}: знаходження $\Delta x_i, i=1,\ldots,n$ за заданими  $\Delta y$, $\delta y$. Якщо $n>1$ маємо одну умову $\Sum_{i=1}^n b_i\Delta x_i<\epsilon$ на багато невідомих $\Delta x_i$. Зазвичай вибирають їх із однієї з умов \[\forall i: b_i\Delta x_i<\epsilon/n \quad \text{або} \quad \forall i: \Delta x_i<\epsilon/B,\quad \text{де} \ws B=\Sum_{i=1}^n b_i.\]

\end{document}