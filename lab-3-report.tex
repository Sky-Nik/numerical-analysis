\documentclass[12pt, a4paper]{article}
\usepackage[utf8]{inputenc}
\usepackage[T2A]{fontenc}
\usepackage[english,ukrainian]{babel}
\usepackage{amsmath, amssymb}
\usepackage{verbatim}

\usepackage[top = 2 cm, left = 1 cm, right = 1 cm, bottom = 2 cm]{geometry}

\usepackage{float, graphicx}
\usepackage{amsthm}
\newtheorem{lemma}{Лема}
\newtheorem*{lemma*}{Лема}
\newtheorem{theorem}{Теорема}
\newtheorem*{theorem*}{Теорема}
\newtheorem{definition}{Визначення}
\newtheorem*{definition*}{Визначення}
\theoremstyle{definition}
\newtheorem{remark}{Зауваження}
\newtheorem*{remark*}{Зауваження}
\newtheorem{example}{Приклад}
\newtheorem*{example*}{Приклад}
\newtheorem{problem}{Задача}
\newtheorem*{problem*}{Задача}
\newtheorem{solution}{Розв'язок}
\newtheorem*{solution*}{Розв'язок}
\newtheorem{corollary}{Наслідок}
\newtheorem*{corollary*}{Наслідок}

\newcommand{\NN}{\mathbb{N}}
\newcommand{\RR}{\mathbb{R}}
\newcommand{\CC}{\mathbb{C}}
\newcommand{\Min}{\displaystyle\min\limits}
\newcommand{\Max}{\displaystyle\max\limits}
\newcommand{\Sup}{\displaystyle\sup\limits}
\newcommand{\Sum}{\displaystyle\sum\limits}
\newcommand{\Prod}{\displaystyle\prod\limits}
\newcommand{\Int}{\displaystyle\int\limits}
\newcommand{\Iint}{\displaystyle\iint\limits}
\newcommand{\Lim}{\displaystyle\lim\limits}

\newcommand*\diff{\mathop{}\!\mathrm{d}}

\renewcommand{\bf}[1]{\textbf{#1}}
\renewcommand{\epsilon}{\varepsilon}
\renewcommand{\phi}{\varphi}

\DeclareMathOperator{\signum}{sign}
\DeclareMathOperator{\diam}{diam}
\DeclareMathOperator{\rang}{rang}
\DeclareMathOperator{\const}{const}
\DeclareMathOperator{\cond}{cond}

\numberwithin{equation}{section}

\begin{document}

\setlength\parindent{0pt}
\allowdisplaybreaks

\begin{center}
\hfill \break
Міністерство освіти та науки України \\
Київський національний університет імені Тараса Шевченка \\ 
Факультет комп'ютерних наук та кібернетики \\
Кафедра обчислювальної математики \\
\vfill 
\large{Звіт до лабораторної роботи №3 на тему: \\ ``Проблема знаходження власних чисел матриці''} \\
\vfill 
\end{center}
\begin{flushright}
Виконав студент групи ОМ-3 \\
Скибицький Нікіта
\end{flushright}
\vfill 
\begin{center}
    Київ, 2018 
\end{center}
\thispagestyle{empty}
 
\newpage

\section{Постановка задачі}

Для матриці \[ a_{i,j} = \begin{cases} \frac{i+j-1}{2n}, & i \ne j \\ n + 10 + \frac{i+j-1}{2n}, & i = j \end{cases} \] порядку $n = 5, 6, \ldots$ знайти 
\begin{enumerate}
	\item $\lambda_{\max} = \Max_i|\lambda_i(A)|$;
	\item $\lambda_{\min} = \Min_i \lambda_i(A)$;
	\item $\hat \lambda_{\max} = \Min_i|\lambda_i(A)|$.
\end{enumerate}
методом скалярних добутків та всі власні значення -- методом обертання Якобі.

\section{Теоретична частина}

\subsection{Степеневий метод}
\begin{enumerate}
	\item \textit{Знаходження} $\lambda_{\max}: |\lambda_1|\equiv\lambda_{\max}> |\lambda_2| \ge |\lambda_3| \ge \ldots$ \\

	Нехай $\vec x^{(0)}$ -- заданий вектор, будемо послідовно обчислювати вектори
	\begin{equation}
		\label{eq:5.2}
		\vec x^{(k+1)} = A \vec x^{(k)}, \quad k = 0, 1, \ldots
	\end{equation}

	Тоді $\vec x^{(k)} = A^k \vec x^{(0)}$. Нехай $\{\vec e_i\}_{i=1}^n$ -- система власних векторів. Представимо $\vec x^{(0)}$ у вигляді: \[ \vec x^{(0)} = \Sum_{i=1}^n c_i \vec e_i. \]

	Оскільки $A \vec e_i = \lambda_i \vec e_i$, то $\vec x^{(k)} = \Sum_{i=1}^n c_i \lambda_i^k \vec e_i$. При великих $k$: $\vec x^{(k)} \approx c_1 \lambda_1^k \vec e_1$. Тому \[ \mu_1^{(k)} = \dfrac{\vec x_m^{(k+1)}}{\vec x_m^{(k)}} = \lambda_1 + O \left( \left| \dfrac{\lambda_2}{\lambda_1} \right|^k \right). \]

	Значить $\mu_1^{(k)} \xrightarrow[k\to\infty]{} \lambda_1$. \\

	Якщо матриця $A = A^T$ симетрична, то існує ортонормована система векторів $(\vec e_i, \vec e_j) = \delta_{i,j}$. Тому

	\begin{multline*}
		\mu_1^{(k)} = \dfrac{(\vec x^{(k+1)}, \vec x^{(k)})}{(\vec x^{(k)}, \vec x^{(k)})} = \dfrac{\left(\Sum_{i=1}^n c_i \lambda_i^(k+1) \vec e_i,\Sum_{j=1}^n c_j \lambda_j^k \vec e_j\right)}{\left(\Sum_{i=1}^n c_i \lambda_i^k \vec e_i,\Sum_{j=1}^n c_j \lambda_j^k \vec e_j\right)} = \dfrac{\Sum_i c_i^2 \lambda_i^{2k+1}}{\Sum_i c_i^2 \lambda_i^{2k}} = \\
		= \dfrac{c_1^2\lambda_1^{2k+1}+c_2^2\lambda_2^{2k+1}+\ldots}{c_1^2\lambda_1^{2k}+c_2^2\lambda_2^{2k}+\ldots} = \lambda_1 + O \left( \left| \dfrac{\lambda_2}{\lambda_1} \right|^{2k} \right) \xrightarrow[k\to\infty]{} \lambda_1.
	\end{multline*}

	Це означає збіжність до максимального за модулем власного значення з квадратичною швидкістю.\\

	Якщо $\lambda_1 > 1$, то при проведенні ітерацій відбувається зріст компонент вектора $\vec x^{(k)}$, що приводить до ``переповнення'' (overflow). Якщо ж $\lambda_1 < 1$, то це приводить до зменшення компонент (underflow). Позбутися негативу такого явища можна нормуючи вектори $\vec x^{(k)}$ на кожній ітерації. \\

	\textbf{Алгоритм} степеневого методу знаходження максимального за модулем власного значення з точністю $\epsilon$ виглядає так:
	\begin{enumerate}
		\item $\vec x^{(0)} \to \vec e_0 = \frac{\vec x^{(0)}}{\|\vec x^{(0)}\|}$;

		\item $\vec x^{(k+1)} = A \vec x^{(k)}$, $\mu_1^{(k)} = (\vec x^{(k+1)}, \vec e^{(k)})$, $\vec e^{(k+1)} = \frac{\vec x^{(k+1)}}{\|\vec x^{(k+1)}\|}$, $k = 0,1,\ldots$;

		\item $|\mu_1^{(k+1)} - \mu_1^{(k)}| \ge \epsilon$ \verb|goto 2|;

		\item $\lambda_1 \approx \mu_1^{(k+1)}$.
	\end{enumerate}

	За цим алгоритмом для симетричної матриці $A^T = A$ швидкість прямування $\mu_1^{(k)}$ до $\lambda_{\max}$ -- квадратична.

	\item \textit{Знаходження} $\lambda_2: |\lambda_1| > |\lambda_2| > |\lambda_3| \ge \ldots$. Нехай $\lambda_1$, $\vec e_1$ відомі. \\

	Якщо $|\lambda_1| > |\lambda_2| > |\lambda_3| \ge \ldots$, то
		\[ \mu_2^{(k)} = \dfrac{\vec x_m^{(k+1) - \lambda_1 \vec x_m^{(k)}}}{\vec x_m^{(k) - \lambda_1 \vec x_m^{(k-1)}}} \xrightarrow[k\to\infty]{} \lambda_2, \quad \text{де } \vec x^{(k+1)} = A \vec x^{(k)}. \]
		$x_m^{(k)}$ -- $m$-та компонента $\vec x^{(k)}$.

	Є алгоритм обчислення $\lambda_2$, $\vec e_2$, використовуючи нормування векторів та скалярні добутки для обчислення $\mu_2^{(k)})$.


	\item \textit{Знаходження мінімального власного числа} $\lambda_{\min}(A) = \min_i |\lambda_i(A)|$.

	Припустимо , що $\lambda_i(A) > 0$ та відоме $\lambda_{\max}$. Розглянемо матрицю $B = \lambda_{\max} E - A$. Маємо \[ \forall i: \lambda_i(B) = \lambda_{\max} - \lambda_i(A). \]

	Тому $\Max_i \lambda_i (B) = \lambda_{\max} - \Min_i \lambda_i(A)$. Звідси $\lambda_{\min}(A) = \lambda_{\max}(A) - \lambda_{\max}(B)$. \\

	Якщо $\exists i: \lambda_i(A) < 0$, то будуємо матрицю $ \overline{A} = \sigma E + A$, $\sigma > 0$, $\overline{A} > 0$ і для неї попередній розгляд дає необхідний результат. Замість $\lambda_{\max}$ в матриці $B$ можна використовувати $\|A\|$. \\

	Ще один спосіб обчислення мінімального власного значення полягає в використання обернених ітерацій:
	\begin{equation}
		\label{eq:5.3}
		A \vec x^{(k+1)} = \vec x^k, \quad k = 0, 1, \ldots
	\end{equation}

	Але цей метод вимагає більшої кількості арифметичних операцій: складність методу на основі формули (\ref{eq:5.2}) $Q = O(n^2)$ , а на основі (\ref{eq:5.3}) -- $Q = O(n^3)$, оскільки треба розв'язувати СЛАР, але збігається метод (\ref{eq:5.3}) швидче.
\end{enumerate}

\subsection{Ітераційний метод обертання}
Це метод розв'язання повної проблеми власних значень для симетричних матриць $A^T = A$. Існує матриця $U$, що приводить матрицю $A$ до діагонального виду:
\begin{equation}
	\label{eq:5.4}
	A = U \Lambda U^T,
\end{equation}
де $\Lambda$ -- діагональна матриця, по діагоналі якої стоять власні значення $\lambda_i$; $U$ -- унітарна матриця, тобто: $U^{-1}=U^T$. \\

З (\ref{eq:5.4}) маємо
\begin{equation}
	\label{eq:5.5}
	\Lambda = U^T \Lambda U,
\end{equation}
Нехай $\exists \tilde U$ -- матриця, така що $\tilde \Lambda = \tilde U^T A \tilde U$ і $\tilde \Lambda = (\tilde \lambda_{i,j})_{i,j=1}^n$, $|\tilde \lambda_{i,j}| < \delta \ll 1$, $i \ne j$. \\

Тоді діагональні елементи мало відрізняються від власних значень \[ |\tilde \lambda_{i,i} - \lambda_i(A) | < \epsilon = \epsilon(\delta). \]

Введемо $t(A) = \Sum_{\substack{i, j = 1 \\ i \ne j}}^n a_{i,j}^2$. З малості величини $t(A)$ витікає, що діагональні елементи малі. По $A = A_0$ за допомогою матриць обертання % U_K
що повертають систему векторів на кут $\phi$, побудуємо послідовність $\{A_k\}$ таку, що $A_k \to \Lambda$ при $k\to\infty$. \\

Матриця обертання $U_k$ є унітарною: $U_k^{-1} = U_k^T$. \\

Послідовно будуємо:
\begin{equation}
	\label{eq:5.6}
	A_{k+1} = U_K^T A_k U_k,
\end{equation}
Процес (\ref{eq:5.6}) називається \textit{монотонним}, якщо: $t(A_{k+1})<t(A_k)$.

Для матриці (\ref{eq:5.6}) виконується:
\begin{equation}
	\label{eq:5.7}
	a_{i,j}^{(k+1)} = a_{i,j}^{(k)} \cos 2 \phi + \dfrac12 (a_{j,j}^{(k)} - a_{i,i}^{(k)}) \sin 2 \phi,
\end{equation}

А також $t(A_{k+1})=t(A_k)-2(a_{i,j}^{(k)})^2$, якщо вибирати $\phi$ з умови $a_{i,j}^{(k+1)} = 0$. \\

Звідси $\phi = \phi_k = \frac12 \arctan(p^{(k)})$, $p^{(k)} = \frac{2a_{i,j}^{(k)}}{a_{i,i}^{(k)}-a_{j,j}^{(k)}}$, де $|a_{i,j}^{(k)}| = \Max_{\substack{m,l \\ m \ne l}} |a_{m,l}^{(k)}|$. Тоді
$t(A_k) \to 0$, $\to \infty$. Чим більше $n$ тим більше ітерацій необхідно для зведення $A$ до $\Lambda$. 

\section{Практична частина}

Покажемо наочні результати для значення $n = 5$ та для заданої точності $\epsilon = 10^{-6}$, хоча запрограмований алгоритм дає змогу отримати результати $\forall n \in \NN$:
\[ A = \begin{pmatrix} 15.1 & 0.2 & 0.3 & 0.4 & 0.5 \\ 0.2 & 15.3 & 0.4 & 0.5 & 0.6 \\ 0.3 & 0.4 & 15.5 & 0.6 & 0.7 \\ 0.4 & 0.5 & 0.6 & 15.7 & 0.8 \\ 0.5 & 0.6 & 0.7 & 0.8 & 15.9 \end{pmatrix}. \]

\subsection{Метод скалярних добутків}

Почнемо з $\vec x^{(0)} = (1, 0, 0, 0, 0)$ щоб довго не думати.

\begin{enumerate}

	\item В процесі розв'язання задачі, були отримані відповідні результати для $\lambda_{\max} = \Max_i |\lambda_i(A)|$ та $\lambda_{\min} = \Min_i |\lambda_i(A)|$:

		\[\lambda_{\max} = 17.686140661634397 \]

	\item Для знаходження $\lambda_{\min}$ була використана допоміжна матриця, $B = \lambda_{\max} \cdot E - A$:

	\[ B = \begin{pmatrix} 2.586140 & -0.2 & -0.3 & -0.4 & -0.5 \\ -0.2 & 2.486140 & -0.4 & -0.5 & -0.6 \\ -0.3 & -0.4 & 2.386140 & -0.6 & -0.7 \\ -0.4 & -0.5 & -0.6 & 2.286140 & -0.8 \\ -0.5 & -0.6 & -0.7 & -0.8 & 2.186140 \end{pmatrix}. \]

	Для якої знаходимо $\lambda_{\min}(A) = \lambda_{\max}(A) - \lambda_{\max}(B)$:

		\[\lambda_{\min} = 17.686140661634397 - 2.8722810782394337 = 14.813859583394963. \]

	\item Для знаходження $\hat \lambda_{\min}$ була використана допоміжна матриця $C = E - \frac{A^2}{\lambda_{\max}^2	}$:

	\[ C = \begin{pmatrix} -11.92254791 & -0.3788277 & -0.55693326 & -0.73503882 & -0.91314438 \\ -0.3788277 & -12.2815861 & -0.74069297 & -0.9216256 & -1.10255823 \\ -0.55693326 & -0.74069297 & -12.64627844 & -1.10821238 & -1.29197208 \\ -0.73503882 & -0.9216256 & -1.10821238 & -13.01662492 & -1.48138593 \\ -0.91314438 & -1.10255823 & -1.29197208 & -1.48138593 & -13.39262555 \end{pmatrix}\]	

	\begin{multline*}\hat \lambda_{\min} = \Min_i |\lambda_i(A)| = \sqrt{\lambda_{\max}^2(A)(1-\lambda_{\max}(C))} = \\ =\sqrt{17.686140661634397^2(1-0.2984311005453412))}= 14.813859583394963. \end{multline*}

\end{enumerate}

\subsection{Метод обертання Якобі}

Враховуючи теоретичний арсенал та застосовуючи заданий алгоритм отримаємо: \[ \lambda(A) = (14.813859, 14.999999, 14.999999, 14.999999, 17.686140)\]
Ітерації виконуємо допоки не буде виконана умова зупинки: \[t(A_k) = \Sum_{\substack{i,j = 1\\i\ne j}}^n a_{i,j}^2 < \epsilon, \quad A_{k+1} = U_k^T \cdot A_k \cdot U_k \]
На останній ітерації: $t(A_N) = 1.1086685175654626\cdot10^{-7}$. Кількість ітерацій $N = 11$

\end{document}