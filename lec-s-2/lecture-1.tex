% cd ..\..\Users\NikitaSkybytskyi\Desktop\c3s2\numerical-analysis
% pdflatex lecture-1.tex && pdflatex lecture-1.tex && del lecture-1.out, lecture-1.log, lecture-1.aux, lecture-1.toc && start lecture-1.pdf

% \documentclass[a4paper, 12pt]{article}
\usepackage[utf8]{inputenc}
\usepackage[english, ukrainian]{babel}
\usepackage{amsmath, amssymb}
\usepackage[top = 1 cm, left = 1 cm, right = 1 cm, bottom = 1 cm]{geometry} 

\usepackage{multicol, graphicx}

\usepackage{amsthm}
\theoremstyle{definition}
\newtheorem{problem}{Задача}
\newtheorem{definition}{Визначення}

\allowdisplaybreaks
\setlength\parindent{0pt}
\pagestyle{empty}

\newcommand{\argmax}{\arg\max}
\newcommand{\argmin}{\arg\min}

\newcommand{\dif}{\mathrm{d}}
\newcommand{\dydx}{\dfrac{\dif y}{\dif x}}
\newcommand{\dxdt}{\dfrac{\dif x}{\dif t}}
\newcommand{\dydt}{\dfrac{\dif y}{\dif t}}

\newcommand{\NN}{\mathbb{N}} 
\newcommand{\ZZ}{\mathbb{Z}}
\newcommand{\QQ}{\mathbb{Q}}
\newcommand{\RR}{\mathbb{R}}
\newcommand{\CC}{\mathbb{C}}

\newcommand{\LaReF}[1]{(\ref{#1})}

\renewcommand{\epsilon}{\varepsilon}
\renewcommand{\phi}{\varphi}

\newcommand{\ws}{\text{ }}

\newcommand{\Max}{\displaystyle\max\limits}
\newcommand{\Min}{\displaystyle\min\limits}
\newcommand{\Sum}{\displaystyle\sum\limits}
\newcommand{\Int}{\displaystyle\int\limits}
\newcommand{\Prod}{\displaystyle\prod\limits}


% \title{Чисельні методи}
% \author{Нікіта Скибицький}
% \date{\today}

% \begin{document}

% \maketitle

% \tableofcontents

\section{Обернена інтерполяція}

\begin{problem}
	Застосовуючи обернену інтерполяцію розв'язати рівняння
	\begin{equation*}
		\ln(x) + x - 2 = 0,
	\end{equation*}
	та оцінити похибку.
\end{problem}

\begin{solution}
	Нескладно переконатися що на проміжку $[1, 2]$ існує і єдиний розв'язок рівняння бо $f(1) = -1 < 0$, $f(2) = \ln 2 > 0$, $f'(x) = \frac1x + 1 > 0$ на $[0, + \infty)$. Знаходячи розділені різниці знаходимо все що хочемо (наближаємо лінійною). \\

	Перевагою поліному Ньютона є те, що ми можемо додати ще один вузел не обов'язково впорядковуючи вузли за зростанням, тоді можна буде не переобчислювати частину (в загальному випадку більшу частину) таблиці розділених різниць. \\

	Нагадаємо, що похибка прямої інтерполяції має вигляд
	\begin{equation*}
		r_n(x) = \frac{f^{(n+1)}(\xi)}{(n + 1)!} \cdot \omega_n(x).
	\end{equation*}

	Знайдемо тепер формулу для похибки для (лінійної) оберненої інтерполяції:
	\begin{equation*}
		|r_1(y)| \le \frac{M_2}{2!} \cdot \omega_1(y) = \frac{M_2}{2!} \cdot (y - y_0) \cdot (y - y_1),
	\end{equation*}
	де $M_2 = \max\limits_{y \in [y_0, y_1]} \left| \frac{\diff^2 x}{\diff y^2} \right|$, де $\frac{\diff x}{\diff y} = \left( \frac{\diff y}{\diff x} \right)$.
\end{solution}

Навчимося також брати $\frac{\diff^n x}{\diff y^n}$ знаючи $\frac{\diff y}{\diff x}$.

\section{Чисельне диференціювання}

Ми розглянемо постановку задачі та ідеологію розв'язання задачі. \\

Нехай задані точки $x_i \in [a, b]$, $i = \overline{0, n}$. У цих точках задані значення, які ми будемо позначати одним з трьох способів: $y_i = f_i = f(x_i)$.\\

Задача полягає у тому, що \textbf{необхідно} знайти значення $k$-ої похідної $f^{(k)}(x)$. \\

Задачу будемо розв'язувати базуючись на інтерполяційному підході: подамо функцію $f(x)$ у вигляді інтерполяційного поліному
\begin{equation}
	\label{eq:1.1}
	f(x) = P_n(x) + r_n(x)
\end{equation}

Тут $P_n$ -- інтерполяційний поліном Ньютона, а $r_n$ -- залишковий член. Подамо залишковмий член у вигляді
\begin{equation}
	\label{eq:1.2}
	r_n(x) = f(x, x_0, \ldots, x_n) \cdot \omega_n(x),
\end{equation}
де
\begin{equation}
	\label{eq:1.3}
	\omega_n(x) = (x - x_0) \cdot \ldots \cdot (x - x_n).
\end{equation}

Тоді, виходячи з \eqref{eq:1.1}, будемо мати
\begin{equation}
	\label{eq:1.4}
	f^{(k)}(x) = P_n^{(k)}(x) + r_n^{(k)}(x), \quad k \le n.
\end{equation}

Зрозуміло, що $k \le n$, бо інакше у нас недостатньо даних для проведення інтерполяції. \\

Таким чином ми за наближене значення $k$-ої похідної будемо брати $k$-у похідну від інтерполяційного поліному:
\begin{equation}
	\label{eq:1.5}
	f^{(n)}(x) \approx P_n^{(k)}(x).
\end{equation}

Оцінимо похибку $r_n^{(k)}(x)$: виходячи з \eqref{eq:1.2} можемо записати
\begin{equation}
	\label{eq:1.6}
	r_n^{(k)}(x) = \Sum_{j=0}^k C_k^i \cdot f^{(j)} (x, x_0, \ldots, x_n) \cdot \omega^{(k - j)}(x).
\end{equation}

Тоді, за формулою зв'язку розділеної різниці і похідної (нагадаємо її для тих, хто не пам'ятає):
\begin{equation*}
	f(x_0, \ldots , x_n) = \frac{f^{(n)}(\xi)}{n!}
\end{equation*}
з \eqref{eq:1.6} маємо
\begin{equation}
	\label{eq:1.7}
	f^{(j)}(x, x_0, \ldots, x_n) = j! \cdot f\left( \underset{j}{\underbrace{x, \ldots, x}}, x_0, \ldots, x_n \right) = \frac{j! \cdot f^{(n + j + 1)}(\xi)}{(n + j + 1!)}
\end{equation}

Підставимо це в \eqref{eq:1.6}:
\begin{equation}
	\label{eq:1.8}
	r_n^{(k)}(x) = \Sum_{j = 0}^k \frac{k!}{(k-j)! \cdot (n+j+1)!} \cdot f^{(n + j + 1)}(\xi_j) \cdot \omega^{k - j}(x).
\end{equation}

Формула \eqref{eq:1.8} дасть нам оцінку похибки:
\begin{equation}
	\label{eq:1.9}
	|r_n^{(k)}(x)| \le M \cdot \Sum_{j = 0}^k \frac{k!}{(k-j)! \cdot (n+j+1)!} \cdot \omega^{(k - j)}(x),
\end{equation}
де
\begin{equation*}
	M = \Max_{[a,b]} \Max_{j = \overline{0, k}} |f^{(n + j + 1)}(x)|
\end{equation*}

Давайте тепер з цієї оцінки отримаємо щось хороше, а то зараз на неї неприємно дивитися. Нехай точки $x_i$ рівновіддалені: $x_i = x_0 + i h$. З'ясуємо який порядок по $h$ матиме похибка. Тоді
\begin{equation*}
	x - x_i \approx O(h),
\end{equation*}
тому
\begin{equation*}
	\omega_n(x) = O(h^{n + 1}),
\end{equation*}
а також 
\begin{equation*}
	\omega_n'(x) = O(h^n).
\end{equation*}
Звідси, за індукцією,
\begin{equation*}
	\omega_n^{(k)}(x) = O(h^{n + 1 - k}).
\end{equation*}

Повернемося до формули \eqref{eq:1.8}. Виходячи з останніх зауважень з'ясуємо який же порядок по $h$ матиме $r_n^{(k)}$. Зрозуміло що найнижчий (тобто найгірший) порядок буде у доданку суми де $j = 0$, а саме:
\begin{equation}
	\label{eq:1.10}
	r_n^{(k)}(x) \approx O(h^{n + 1 - k}).
\end{equation}

Розгялнемо ще одну особливість, таку як точки підвищеної точності.

\subsection{Точки підвищеної точності}

Давайте тепер формулу \eqref{eq:1.8} запишемо у двох частинах, окремо для $j = 0$:
\begin{multline}
	\label{eq:1.11}
	r_n^{(k)}(x) = \frac{f^{(n + 1)}(\xi_0)}{(n + 1)!} \cdot \omega_n^{(k)}(x) + \\ + \Sum_{j = 1}^k \frac{k!}{(k-j)! \cdot (n+j+1)!} \cdot f^{(n + j + 1)}(\xi_j) \cdot \omega^{k - j}(x)
\end{multline}

З формули \eqref{eq:1.11} бачимо наступне: перший доданок має порядок $O(h^{n + 1 - k})$, а для другого додатку ``найгіршим'' випадком буде $j = 1$, причому його порядком буде $O(h^{n + 2 - k})$, звідки випливає таке спостержеення. \\

Нехай $\bar x: \omega_n^{(k)}(\bar x) = 0$, тоді 
\begin{equation}
	\label{eq:1.12}
	r_n^{(k)}(\bar x) \approx O({h^{n + k - 2}}).
\end{equation}

Такі точки називаються \textit{точками підвищеної точності}.

\begin{example}
	Побудувати формулу чисельного диференціювання для $n = 1$, $k = 1$ і знайти точки підвищеної точності. \\

	Оскільки $n = 1$, то
	\begin{equation}
		\label{eq:1.14}
		P_1(x) = f_0 + (x - x_0) \cdot \frac{f_1 - f_0}{h}
	\end{equation}
	де $h = x_1 - x_0$. \\

	Тоді
	\begin{equation}
		\label{eq:1.15}
		P_1'(x) = \frac{f_1 - f_0}{h}.
	\end{equation}

	Таким чином для $x \in [x_0, x_1]$ похідну можна знаходити за формулою \eqref{eq:1.15}. \\

	Знайдемо тепер похибку. За формулою \eqref{eq:1.8} маємо
	\begin{multline}
		\label{eq:1.16}
		r_1'(x) = \frac{f''(\xi_0)}{2!} \cdot \omega_1'(x) + \frac{f'''(\xi_1)}{3!} \cdot \omega_1(x) = \\ = \frac{f''(\xi_0)}{2} \cdot (2x - x_0 - x_1) + \frac{f'''(\xi_1)}{6} \cdot (x - x_0) \cdot (x - x_1).
	\end{multline}

	Отже порядок формули буде $O(h)$, тобто $r_1'(x) \approx O(h)$, $f \in C^{(2)}$.\\

	Знайдемо точки підвищеної точності. Нехай $\omega_1'(x) = 0$, тоді
	\begin{equation}
		\label{eq:1.17}
		\bar x = \frac{x_0 + x_1}{2},
	\end{equation}
	і похибка має порядок $O(h^2)$, 
	\begin{equation}
		\label{eq:1.18}
		r_1'(\bar x) = - \frac{h^2}{24} \cdot f'''(\xi_1),
	\end{equation}
	або ж
	\begin{equation}
		\label{eq:1.19}
		|r_1'(\bar x)| \le \frac{M_3 \cdot h^2}{24},
	\end{equation}
	де $M_3 = \Max_{[x_0, x_1]} |f'''(x)|$.
\end{example}

Таким чином, похідну на проміжку $x \in [x_0, x_1]$ можна знаходити ща формулою \eqref{eq:1.15}. Переведемо тепер цю формулу на проміжок $x \in [x_{i - 1}, x_i]$, отримаємо
\begin{equation}
	\label{eq:1.20}
	f'(x) \approx \frac{f_i - f_{i - 1}}{h} \overset{\triangle}{=} f_{\bar x, i}.
\end{equation}
Її називають \textit{різницевою похідною назад в точці $x_i$}, або ще \textit{лівою різницевою похідною}. \\

Якщо ж $x \in [x_i, x_{i+1}]$, то
\begin{equation}
	\label{eq:1.21}
	f'(x) \approx \frac{f_{i + 1} - f_i}{h} \overset{\triangle}{=} f_{x, i}.
\end{equation}
Її називають \textit{різницевою похідною вперед в точці $x_i$}, або ще \textit{правою різницевою похідною}. 

\begin{problem}
	Побудувати формулу чисельного диференціювання для $n = 2$, $k = 1$ для випадку рівновіддалених вузлів. Знайти точки підвищенної точності.
\end{problem}

\begin{remark*}
	Для формул \eqref{eq:1.20} та \eqref{eq:1.21} похибка має порядок $O(h)$, $r_1'(x) \approx O(h)$.
\end{remark*}

\begin{problem}
	Нехай тепер
	\begin{enumerate}
		\item[а)] \begin{equation}
			\label{eq:1.22}
			f'(x_i) \approx f_{\bar x, i}
		\end{equation}
		\item[б)] \begin{equation}
			\label{eq:1.23}
			f'(x_i) \approx f_{x, i}
		\end{equation}
		\item[в)] \begin{equation}
			\label{eq:1.24}
			f'(x_i) \approx \frac{f_{i + 1} - f_{i - 1}}{2 h}, \quad x \in [x_{i - 1}, x_{i + 1}].
		\end{equation}
	\end{enumerate}

	Визначити похибку чисельного диференціювання для формул \eqref{eq:1.22}-\eqref{eq:1.24}.
\end{problem}
\begin{hint}
	Розв'язок можна безпосередньо отримати з формули \eqref{eq:1.16} якщо підставити замість $x$ потрібну точку ($x_0$, $x_1$, або $\frac{x_0+x_1}{2}$ відповідно).
\end{hint}

Розгялнемо ще один підхід як можна знайти похибку чисельного диференціювання.

\subsection{Похибка через ряд Тейлора}

\begin{example}
	Для рівновіддалених точок побудувати формулу чисельного диференціювання для $n = 2$, $k = 2$ та визначити похибку цієї формули за допомогоб розвинення в ряд Тейлора:
	\begin{equation*}
		P_2(x) = f_0 + (x - x_0) \cdot \frac{f_1 - f_0}{2} + (x - x_0) \cdot (x - x_1) \cdot \frac{f_2 - 2 f_1 + f_2}{2 h^2}.
	\end{equation*}

	Тоді будемо мати, що 
	\begin{equation}
		\label{eq:1.25}
		P_2''(x) = \frac{f_2 - 2 f_1 + f_0}{h^2}.
	\end{equation}

	Перепишемо тепер формулу \eqref{eq:1.25} для довільного вузла $x$:
	\begin{equation}
		\label{eq:1.27}
		P_2''(x) = \frac{f_{i + 1} - 2 f_i + f_{i - 1}}{h^2}, \quad x \in [x_{i - 1}, x_{i + 1}] \overset{\triangle}{=} f_{\bar x x, i}.
	\end{equation}	

	Тоді, для формули \eqref{eq:1.27} матимемо
	\begin{multline*}
		r_2''(x) = f''(x) - \frac{f_{i + 1} - 2 f_i + f_{i - 1}}{h^2} = \\ = f''(x) - \frac{1}{h^2} \left( f_i + h f_i' + \frac{h^2}{2} f_i'' + \frac{h^3 f_i'''}{6} + \frac{h^4}{24} f_i^{(4)} - 2f_i \right. + \\ \left. + f_i - h f_i' + \frac{h^2}{2} f_i'' - \frac{h^3}{6} f_i''  + \frac{h^4}{24} f_i^{(4)} \right) + O(h^4) = \\ = f''(x) - f''(x_i) - \frac{h^2}{12} f^{(4)}(x_i) + O(h^4).
	\end{multline*}

	Таким чином, 
	\begin{equation*}
		r_2''(x_i) = -\frac{h^2}{12} f^{(4)} + O(h^4) \approx O(h^2).
	\end{equation*}
\end{example}

Також зауважимо, що \eqref{eq:1.27} визначає \textit{другу різницеву похідну за трьома точками}
 і позначається $f_{\bar x x,i}$.

\begin{problem}
	За допомогою формули чисельного диференціювання для $n = 2$, $k = 1$ знайти похідну в точці $x_i + \frac{h}{2}$.
\end{problem}

% \end{document}
