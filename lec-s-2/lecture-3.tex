% cd ..\..\Users\NikitaSkybytskyi\Desktop\c3s2\numerical-analysis
% pdflatex lecture-3.tex && pdflatex lecture-3.tex && del lecture-3.out, lecture-3.log, lecture-3.aux, lecture-3.toc && start lecture-3.pdf

% \documentclass[a4paper, 12pt]{article}
\usepackage[utf8]{inputenc}
\usepackage[english, ukrainian]{babel}
\usepackage{amsmath, amssymb}
\usepackage[top = 1 cm, left = 1 cm, right = 1 cm, bottom = 1 cm]{geometry} 

\usepackage{multicol, graphicx}

\usepackage{amsthm}
\theoremstyle{definition}
\newtheorem{problem}{Задача}
\newtheorem{definition}{Визначення}

\allowdisplaybreaks
\setlength\parindent{0pt}
\pagestyle{empty}

\newcommand{\argmax}{\arg\max}
\newcommand{\argmin}{\arg\min}

\newcommand{\dif}{\mathrm{d}}
\newcommand{\dydx}{\dfrac{\dif y}{\dif x}}
\newcommand{\dxdt}{\dfrac{\dif x}{\dif t}}
\newcommand{\dydt}{\dfrac{\dif y}{\dif t}}

\newcommand{\NN}{\mathbb{N}} 
\newcommand{\ZZ}{\mathbb{Z}}
\newcommand{\QQ}{\mathbb{Q}}
\newcommand{\RR}{\mathbb{R}}
\newcommand{\CC}{\mathbb{C}}

\newcommand{\LaReF}[1]{(\ref{#1})}

\renewcommand{\epsilon}{\varepsilon}
\renewcommand{\phi}{\varphi}

\newcommand{\ws}{\text{ }}

\newcommand{\Max}{\displaystyle\max\limits}
\newcommand{\Min}{\displaystyle\min\limits}
\newcommand{\Sum}{\displaystyle\sum\limits}
\newcommand{\Int}{\displaystyle\int\limits}
\newcommand{\Prod}{\displaystyle\prod\limits}


% \title{Чисельні методи}
% \author{Нікіта Скибицький}
% \date{\today}

% \begin{document}

% \maketitle

% \tableofcontents

\subsection{Елемент найкращого середньоквадратичного наближення}

Ми зараз будемо розглядати елементи найкращого наближення в різних конкретних гілбертових просторах, але загальну теорію сформулюємо для абстрактного гільбертового простору $H$. \\

Отже, нехай $R = H$, тоді $\exists !$ ЕНН. Тоді
\begin{equation}
	\label{eq:2.12}
	\Delta(f, \Phi) = \| f - \Phi \|.
\end{equation}
А наша формальна задача набуває вигляду: знайти
\begin{equation}
	\label{eq:2.14}
	\inf_{\Phi \in M_n} \Delta(f, \Phi) = \Delta(f).
\end{equation}

Тоді ЕНН $\Phi_0$ має вигляд:
\begin{equation}
	\label{eq:2.15}
	\Phi = \sum_{i = 0}^n c_i \cdot \phi_i,
\end{equation}
де $\phi_i \in M_n$.

\begin{equation}
	\label{eq:2.16}
	\left\| f - \sum_{i = 0}^n c_i \cdot \phi_i \right\|^2 = (f, f) + \sum_{i, j = 0}^n c_i c_j \cdot (\phi_i, \phi_j) - 2 \sum_{i = 0}^n c_i \cdot (f, \phi_i).
\end{equation}

Для ЕНН $\Phi_0$ з \eqref{eq:2.8} маємо:
\begin{equation*}
	(f - \Phi_0, \Phi = 0), \quad \forall \Phi \in M_n,
\end{equation*}
а тоді
\begin{equation}
	\label{eq:2.17}
	(f - \Phi_0, \phi_i) = 0, \quad i = \overline{0, n}.
\end{equation}

Підставляючи в \eqref{eq:2.17} вираз \eqref{eq:2.15} будемо мати наступне:
\begin{equation}
	\label{eq:2.18}
	\sum_{j = 0}^n c_j \cdot (\phi_j, \phi_i) = (f, \phi_i), \quad i = \overline{0, n}.
\end{equation}
Це є СЛАР відносно невідомих коефіцієнтів $c_j$. \\

Якщо система функцій $\{\phi_i\}$ є лінійно-незалежною, то матриця $G = (g_{i, j})_{i, j = \overline{0, n}}$, то $det G \ne 0$ і система \eqref{eq:2.18} матиме єдиний розв'язок. \\

Система \eqref{eq:2.18} має симетричну матрицю, тому доцільно розв'язувати \eqref{eq:2.18} методом квадратних коренів, тобто можемо записати її ще у такому вигляді:
\begin{equation}
	\label{eq:2.18_prime}
	G \overline{c} = \overline{F},
\end{equation}
де
\begin{equation*}
	\overline{F} = \begin{pmatrix} (f, \phi_0) \\ (f, \phi_1) \\ \vdots \\ (f, \phi_n) \end{pmatrix}	
\end{equation*}

Відхилення ЕНН у цьому випадку визначатиметься за формулою:
\begin{equation}
	\label{eq:2.19}
	\Delta(f) = \| f - \Phi_0\|.	
\end{equation}

\begin{remark*}
	Систему вигляду \eqref{eq:2.18} можна отримати виходячи з \eqref{2.16}. Справді, якщо ми візьмемо праву частину \eqref{eq:2.16} і позначимо її через функціонал від $c$, $\Phi(c)$, і запишемо $\frac{\partial \Phi}{\partial c_i} = 0$.
\end{remark*}

Таким чином ми побудували алгоритм і нібито все добре, але нагадаємо таку річ як число обумовленості:
\begin{equation*}
	cond(A) = \|A^{-1}\| \cdot \|A\|.
\end{equation*}

Ситуація наступна: нехай $\{\phi_i\} = \{x^i\}_{i = 0}^\infty$, тоді якщо ми візьмемо
\begin{equation*}
	\Phi_n = \sum_{i = 0}^n c_i \cdot x^i
\end{equation*}
і будемо шукати ЕНН у такому вигляді, то для $H = L_2([0, 1])$. У цьому випадку 
\begin{equation*}
	g_{i, j} = \int_0^1 x^i \cdot x^j \diff x = \frac{1}{i + j + 1}, \quad i, j = \overline{0, n},
\end{equation*}
а тоді $cond(G) \approx 10^8$ при $n = 7$, $cond(G) \approx 10^9$ при $n = 8$ і так далі, тобто дуже швидко росте, тобто система \eqref{eq:2.18} буде дуже погано обумовленою.

\begin{theorem}[Мюнца]
	Система функцій $1, \{x^{n_i}\}$, $0 < n_1 < n_2 < \ldots $ буде повною якщо ряд
	\begin{equation*}
		\sum_{i=0}^\infty \frac{1}{n_i}
	\end{equation*}
	буде розбігатися.
\end{theorem}

Якщо ж у \eqref{eq:2.18} ми розглянемо ортогональну систему функцій $\{\phi_i\}$ простору $H$, то матриця $G$ системи \eqref{eq:2.18} буде діагональною, а $c_i$ можна буде знайти за формулами
\begin{equation}
	\label{eq:2.20}
	c_i = \frac{(f, \phi_i)}{(\phi_i, \phi_i)}, \quad i = \overline{0, n},
\end{equation}
а $\Phi_0$ можна буде як і раніше знайти за формулою \eqref{eq:2.15}:
\begin{equation*}
	\Phi = \sum_{i = 0}^n c_i \cdot \phi_i.
\end{equation*}

Нагадаємо умови ортогональності:
\begin{equation*}
	(\phi_i, \phi_j) = \begin{cases} 0, & i \ne j, \\ \ne 0, & i = j. \end{cases}
\end{equation*}

У цьому випадку з \eqref{eq:2.16} будемо мати, що
\begin{equation}
	\label{eq:21}
	\| f - \Phi\|^2 = \|f\|^2 - \sum_{i = 0}^n c_i^2 \cdot \|\phi_i\|^2.
\end{equation}

Якщо ж $\{\phi\}$ -- ортонормована ситсема функцій, тобто
\begin{equation*}
	(\phi_i, \phi_j) = \delta_{i, j} = \begin{cases} 0, & i \ne j, \\ 1, & i = j. \end{cases}
\end{equation*}

У цьому випадку розв'язок системи \eqref{eq:2.18} матиме вигляд
\begin{equation}
	\label{eq:2.22}
	c_i = (f, \phi_i), \quad i = \overline{0, n},
\end{equation}

\begin{equation}
	\label{eq:2.23}
	\Phi_0 = \sum_{i = 0}^n c_i \cdot \phi_i,
\end{equation}

\begin{equation}
	\label{eq:2.24}
	\Delta^2 (f) = \|f - \Phi_0\|^2 = \|f\|^2 - \sum_{i = 0}^n c_i^2.
\end{equation}

Можемо записати розвинення функції $f$ в ряд Фур'є:
\begin{equation}
	\label{eq:2.25}
	f = \sum_{i = 0}^\infty (f, \phi_i) \cdot \phi_i,
\end{equation}

\begin{equation*}
	\|f\|^2 = \sum_{i = 0}^\infty c_i^2.
\end{equation*}

Тоді виходячи з \eqref{eq:2.24} будемо мати, що
\begin{equation*}
	\Delta^2(f) = \| f - \Phi_0\|^2 = \Sum_{i = n + 1}^\infty c_i^2 \xrightarrow[n \to \infty]{} 0.
\end{equation*}

\begin{remark*}
	Якщо система функцій $\{\phi_i\}$ ортогональна з ваговим коефіцієнтом $\rho$, тобто 
	\begin{equation*}
		(\phi_i, \phi_j)_\rho = (\rho \phi_i, \phi_j) = \begin{cases} 0, & i \ne j, \\ \ne 0, & i = j, \end{cases}
	\end{equation*}
	то у цьому випадку формула \eqref{eq:2.20} набуде вигляду
	\begin{equation}
		\label{eq:2.27}
		c_i = \frac{(f, \phi_i)_\rho}{(\phi_i, \phi_i)_\rho}, \quad i = \overline{0, n}.
	\end{equation}
\end{remark*}

\subsubsection{Системи ортонормованих функцій}

\begin{enumerate}
	\item На $[-1, 1]$ з $\rho = 1$ є поліноми Лежандра.
	\item На $[-1, 1]$ з $\rho = \frac{1}{\sqrt{1 - x^2}}$ є поліноми Чебишева.
	\item На $[0, \infty)$ -- поліноми Ерміта.
	\item поліноми Якобі
	\item ...
\end{enumerate}

% \subsubsection{}

% $H - L_2(a, b)$, у ньому $(u, v) = \frac{1}{b - a} \int_a^b u(x) v(x) \diff x$.

\subsubsection{На дискретній множині точок}

$H - \ell_2$, простір сіткових функцій, у ньому $(u, v) = \frac{1}{N} \sum_{i = 1}^N u_i v_i$,

тоді $Q_n(x) = \sum_{j = 0}^n a_j \cdot x^j$. \\

Система функцій $\{\phi_i\} = \{x^i\}$ буде лінійно незалежною якщо $n \le N - 1$. У свою чергу при $n \ge N$ поліном $Q(x) = (x - x_1) \cdot \ldots \cdot (x - x_n)$ може обертатися в 0 на всій сітці якщо в якості $x_i$ взяти вузли сітки оскільки $n \ge N$.

\subsubsection{Для періодичних функцій}

Нехай $x \in [0, 2 \pi)$ і розглянемо функцію $\{\phi_k\} = \{1, \cos x, \sin x, \ldots, \cos kx, \sin k x \}$, тоді тригонометричним поліномом степеню $m$ назвемо
\begin{equation}
	\label{eq:2.28}
	T_m(x) = \sum_{k = 0}^m (a_k \cdot \cos k x + b_k \cdot \sin k x).
\end{equation}

\begin{align}
	\label{eq:2.29}
	(\sin k x, \sin m x) &= \begin{cases} 0, & k = m \\ \pi, & k \ne m, \end{cases} \\
	(\sin k x, \cos m x) &= 0, \nonumber \\
	(\cos k x, \cos m x) &= \begin{cases} 2 \pi , & k = m = 0 \\ \pi, & k = m \ne 0 \\ 0, & k \ne m. \end{cases} \nonumber 
\end{align}

\begin{equation}
	\label{eq:2.30}
	a_0 = \frac{1}{2 \pi} \int_0^{2 \pi} f(x) \diff x,
\end{equation}

\begin{align}
	\label{eq:2.31}
	a_k &= \frac{1}{\pi} \int_0^{2\pi} f(x) \cdot \cos (k x) \diff x, \quad k = \overline{1, m}, \\
	\label{eq:2.32}
	b_k &= \frac{1}{\pi} \int_0^{2\pi} f(x) \cdot \sin (k x) \diff x, \quad k = \overline{1, m}.
\end{align}

\begin{remark*}
	Якщо функція періодична не на $[0, 2 \pi]$, то потрібно зводити або функцію на відповіний проміжок, або систему функцій.
\end{remark*}

\begin{remark*}
	Якщо функція $f$ парна відносно точки $\pi$, то $b_k = 0$, а
	\begin{equation*}
		a_k = \frac{2}{\pi} \int_0^{\pi} f(x) \cdot \cos (k x) \diff x, \quad k = \overline{1, m},
	\end{equation*}
	а якщо непарна, то $a_k = 0$, а
	\begin{equation*}
		b_k = \frac{2}{\pi} \int_0^{\pi} f(x) \cdot \sin (k x) \diff x, \quad k = \overline{1, m}.
	\end{equation*}
\end{remark*}

\begin{problem}
	Показати, що тригонометрчина система функцій до $\sin m x, \cos m x$ на множині рівновіддалених точок $x_i = i \cdot \frac{2 \pi}{n}$, $i = \overline{1, n}$, $x_i \in (0, 2 \pi]$ є ортогональною, та ортонормувати її.
\end{problem}

\begin{remark*}
	У нас степінь полінома $2 m \le n - 1$, а тоді тригонометричний поліном на дискретній множині точок ми можемо записати у вигляді \eqref{eq:2.28} з коефіцієнтами
	\begin{align*}
		a_0 &= \frac1n \sum_{j = 1}^n f(x_j), \\
		a_k &= \frac2n \sum_{j = 1}^n f(x_j) \cdot \cos(k x_j), \\
		b_k &= \frac2n \sum_{j = 1}^n f(x_j) \cdot \sin(k x_j).
	\end{align*}
\end{remark*}

% \end{document}
